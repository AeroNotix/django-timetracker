% Generated by Sphinx.
\def\sphinxdocclass{report}
\documentclass[letterpaper,10pt,english]{sphinxmanual}
\usepackage[utf8]{inputenc}
\DeclareUnicodeCharacter{00A0}{\nobreakspace}
\usepackage[T1]{fontenc}
\usepackage{babel}
\usepackage{times}
\usepackage[Bjarne]{fncychap}
\usepackage{longtable}
\usepackage{sphinx}
\usepackage{multirow}


\title{Timetracker Documentation}
\date{May 20, 2012}
\release{1}
\author{Aaron France}
\newcommand{\sphinxlogo}{}
\renewcommand{\releasename}{Release}
\makeindex

\makeatletter
\def\PYG@reset{\let\PYG@it=\relax \let\PYG@bf=\relax%
    \let\PYG@ul=\relax \let\PYG@tc=\relax%
    \let\PYG@bc=\relax \let\PYG@ff=\relax}
\def\PYG@tok#1{\csname PYG@tok@#1\endcsname}
\def\PYG@toks#1+{\ifx\relax#1\empty\else%
    \PYG@tok{#1}\expandafter\PYG@toks\fi}
\def\PYG@do#1{\PYG@bc{\PYG@tc{\PYG@ul{%
    \PYG@it{\PYG@bf{\PYG@ff{#1}}}}}}}
\def\PYG#1#2{\PYG@reset\PYG@toks#1+\relax+\PYG@do{#2}}

\expandafter\def\csname PYG@tok@gd\endcsname{\def\PYG@tc##1{\textcolor[rgb]{0.63,0.00,0.00}{##1}}}
\expandafter\def\csname PYG@tok@gu\endcsname{\let\PYG@bf=\textbf\def\PYG@tc##1{\textcolor[rgb]{0.50,0.00,0.50}{##1}}}
\expandafter\def\csname PYG@tok@gt\endcsname{\def\PYG@tc##1{\textcolor[rgb]{0.00,0.25,0.82}{##1}}}
\expandafter\def\csname PYG@tok@gs\endcsname{\let\PYG@bf=\textbf}
\expandafter\def\csname PYG@tok@gr\endcsname{\def\PYG@tc##1{\textcolor[rgb]{1.00,0.00,0.00}{##1}}}
\expandafter\def\csname PYG@tok@cm\endcsname{\let\PYG@it=\textit\def\PYG@tc##1{\textcolor[rgb]{0.25,0.50,0.56}{##1}}}
\expandafter\def\csname PYG@tok@vg\endcsname{\def\PYG@tc##1{\textcolor[rgb]{0.73,0.38,0.84}{##1}}}
\expandafter\def\csname PYG@tok@m\endcsname{\def\PYG@tc##1{\textcolor[rgb]{0.13,0.50,0.31}{##1}}}
\expandafter\def\csname PYG@tok@mh\endcsname{\def\PYG@tc##1{\textcolor[rgb]{0.13,0.50,0.31}{##1}}}
\expandafter\def\csname PYG@tok@cs\endcsname{\def\PYG@tc##1{\textcolor[rgb]{0.25,0.50,0.56}{##1}}\def\PYG@bc##1{\setlength{\fboxsep}{0pt}\colorbox[rgb]{1.00,0.94,0.94}{\strut ##1}}}
\expandafter\def\csname PYG@tok@ge\endcsname{\let\PYG@it=\textit}
\expandafter\def\csname PYG@tok@vc\endcsname{\def\PYG@tc##1{\textcolor[rgb]{0.73,0.38,0.84}{##1}}}
\expandafter\def\csname PYG@tok@il\endcsname{\def\PYG@tc##1{\textcolor[rgb]{0.13,0.50,0.31}{##1}}}
\expandafter\def\csname PYG@tok@go\endcsname{\def\PYG@tc##1{\textcolor[rgb]{0.19,0.19,0.19}{##1}}}
\expandafter\def\csname PYG@tok@cp\endcsname{\def\PYG@tc##1{\textcolor[rgb]{0.00,0.44,0.13}{##1}}}
\expandafter\def\csname PYG@tok@gi\endcsname{\def\PYG@tc##1{\textcolor[rgb]{0.00,0.63,0.00}{##1}}}
\expandafter\def\csname PYG@tok@gh\endcsname{\let\PYG@bf=\textbf\def\PYG@tc##1{\textcolor[rgb]{0.00,0.00,0.50}{##1}}}
\expandafter\def\csname PYG@tok@ni\endcsname{\let\PYG@bf=\textbf\def\PYG@tc##1{\textcolor[rgb]{0.84,0.33,0.22}{##1}}}
\expandafter\def\csname PYG@tok@nl\endcsname{\let\PYG@bf=\textbf\def\PYG@tc##1{\textcolor[rgb]{0.00,0.13,0.44}{##1}}}
\expandafter\def\csname PYG@tok@nn\endcsname{\let\PYG@bf=\textbf\def\PYG@tc##1{\textcolor[rgb]{0.05,0.52,0.71}{##1}}}
\expandafter\def\csname PYG@tok@no\endcsname{\def\PYG@tc##1{\textcolor[rgb]{0.38,0.68,0.84}{##1}}}
\expandafter\def\csname PYG@tok@na\endcsname{\def\PYG@tc##1{\textcolor[rgb]{0.25,0.44,0.63}{##1}}}
\expandafter\def\csname PYG@tok@nb\endcsname{\def\PYG@tc##1{\textcolor[rgb]{0.00,0.44,0.13}{##1}}}
\expandafter\def\csname PYG@tok@nc\endcsname{\let\PYG@bf=\textbf\def\PYG@tc##1{\textcolor[rgb]{0.05,0.52,0.71}{##1}}}
\expandafter\def\csname PYG@tok@nd\endcsname{\let\PYG@bf=\textbf\def\PYG@tc##1{\textcolor[rgb]{0.33,0.33,0.33}{##1}}}
\expandafter\def\csname PYG@tok@ne\endcsname{\def\PYG@tc##1{\textcolor[rgb]{0.00,0.44,0.13}{##1}}}
\expandafter\def\csname PYG@tok@nf\endcsname{\def\PYG@tc##1{\textcolor[rgb]{0.02,0.16,0.49}{##1}}}
\expandafter\def\csname PYG@tok@si\endcsname{\let\PYG@it=\textit\def\PYG@tc##1{\textcolor[rgb]{0.44,0.63,0.82}{##1}}}
\expandafter\def\csname PYG@tok@s2\endcsname{\def\PYG@tc##1{\textcolor[rgb]{0.25,0.44,0.63}{##1}}}
\expandafter\def\csname PYG@tok@vi\endcsname{\def\PYG@tc##1{\textcolor[rgb]{0.73,0.38,0.84}{##1}}}
\expandafter\def\csname PYG@tok@nt\endcsname{\let\PYG@bf=\textbf\def\PYG@tc##1{\textcolor[rgb]{0.02,0.16,0.45}{##1}}}
\expandafter\def\csname PYG@tok@nv\endcsname{\def\PYG@tc##1{\textcolor[rgb]{0.73,0.38,0.84}{##1}}}
\expandafter\def\csname PYG@tok@s1\endcsname{\def\PYG@tc##1{\textcolor[rgb]{0.25,0.44,0.63}{##1}}}
\expandafter\def\csname PYG@tok@gp\endcsname{\let\PYG@bf=\textbf\def\PYG@tc##1{\textcolor[rgb]{0.78,0.36,0.04}{##1}}}
\expandafter\def\csname PYG@tok@sh\endcsname{\def\PYG@tc##1{\textcolor[rgb]{0.25,0.44,0.63}{##1}}}
\expandafter\def\csname PYG@tok@ow\endcsname{\let\PYG@bf=\textbf\def\PYG@tc##1{\textcolor[rgb]{0.00,0.44,0.13}{##1}}}
\expandafter\def\csname PYG@tok@sx\endcsname{\def\PYG@tc##1{\textcolor[rgb]{0.78,0.36,0.04}{##1}}}
\expandafter\def\csname PYG@tok@bp\endcsname{\def\PYG@tc##1{\textcolor[rgb]{0.00,0.44,0.13}{##1}}}
\expandafter\def\csname PYG@tok@c1\endcsname{\let\PYG@it=\textit\def\PYG@tc##1{\textcolor[rgb]{0.25,0.50,0.56}{##1}}}
\expandafter\def\csname PYG@tok@kc\endcsname{\let\PYG@bf=\textbf\def\PYG@tc##1{\textcolor[rgb]{0.00,0.44,0.13}{##1}}}
\expandafter\def\csname PYG@tok@c\endcsname{\let\PYG@it=\textit\def\PYG@tc##1{\textcolor[rgb]{0.25,0.50,0.56}{##1}}}
\expandafter\def\csname PYG@tok@mf\endcsname{\def\PYG@tc##1{\textcolor[rgb]{0.13,0.50,0.31}{##1}}}
\expandafter\def\csname PYG@tok@err\endcsname{\def\PYG@bc##1{\setlength{\fboxsep}{0pt}\fcolorbox[rgb]{1.00,0.00,0.00}{1,1,1}{\strut ##1}}}
\expandafter\def\csname PYG@tok@kd\endcsname{\let\PYG@bf=\textbf\def\PYG@tc##1{\textcolor[rgb]{0.00,0.44,0.13}{##1}}}
\expandafter\def\csname PYG@tok@ss\endcsname{\def\PYG@tc##1{\textcolor[rgb]{0.32,0.47,0.09}{##1}}}
\expandafter\def\csname PYG@tok@sr\endcsname{\def\PYG@tc##1{\textcolor[rgb]{0.14,0.33,0.53}{##1}}}
\expandafter\def\csname PYG@tok@mo\endcsname{\def\PYG@tc##1{\textcolor[rgb]{0.13,0.50,0.31}{##1}}}
\expandafter\def\csname PYG@tok@mi\endcsname{\def\PYG@tc##1{\textcolor[rgb]{0.13,0.50,0.31}{##1}}}
\expandafter\def\csname PYG@tok@kn\endcsname{\let\PYG@bf=\textbf\def\PYG@tc##1{\textcolor[rgb]{0.00,0.44,0.13}{##1}}}
\expandafter\def\csname PYG@tok@o\endcsname{\def\PYG@tc##1{\textcolor[rgb]{0.40,0.40,0.40}{##1}}}
\expandafter\def\csname PYG@tok@kr\endcsname{\let\PYG@bf=\textbf\def\PYG@tc##1{\textcolor[rgb]{0.00,0.44,0.13}{##1}}}
\expandafter\def\csname PYG@tok@s\endcsname{\def\PYG@tc##1{\textcolor[rgb]{0.25,0.44,0.63}{##1}}}
\expandafter\def\csname PYG@tok@kp\endcsname{\def\PYG@tc##1{\textcolor[rgb]{0.00,0.44,0.13}{##1}}}
\expandafter\def\csname PYG@tok@w\endcsname{\def\PYG@tc##1{\textcolor[rgb]{0.73,0.73,0.73}{##1}}}
\expandafter\def\csname PYG@tok@kt\endcsname{\def\PYG@tc##1{\textcolor[rgb]{0.56,0.13,0.00}{##1}}}
\expandafter\def\csname PYG@tok@sc\endcsname{\def\PYG@tc##1{\textcolor[rgb]{0.25,0.44,0.63}{##1}}}
\expandafter\def\csname PYG@tok@sb\endcsname{\def\PYG@tc##1{\textcolor[rgb]{0.25,0.44,0.63}{##1}}}
\expandafter\def\csname PYG@tok@k\endcsname{\let\PYG@bf=\textbf\def\PYG@tc##1{\textcolor[rgb]{0.00,0.44,0.13}{##1}}}
\expandafter\def\csname PYG@tok@se\endcsname{\let\PYG@bf=\textbf\def\PYG@tc##1{\textcolor[rgb]{0.25,0.44,0.63}{##1}}}
\expandafter\def\csname PYG@tok@sd\endcsname{\let\PYG@it=\textit\def\PYG@tc##1{\textcolor[rgb]{0.25,0.44,0.63}{##1}}}

\def\PYGZbs{\char`\\}
\def\PYGZus{\char`\_}
\def\PYGZob{\char`\{}
\def\PYGZcb{\char`\}}
\def\PYGZca{\char`\^}
\def\PYGZam{\char`\&}
\def\PYGZlt{\char`\<}
\def\PYGZgt{\char`\>}
\def\PYGZsh{\char`\#}
\def\PYGZpc{\char`\%}
\def\PYGZdl{\char`\$}
\def\PYGZti{\char`\~}
% for compatibility with earlier versions
\def\PYGZat{@}
\def\PYGZlb{[}
\def\PYGZrb{]}
\makeatother

\begin{document}

\maketitle
\tableofcontents
\phantomsection\label{index::doc}



\chapter{Contents:}
\label{index:contents}\label{index:welcome-to-timetracker-s-documentation}

\section{Dependencies}
\label{deps:dependencies}\label{deps::doc}

\subsection{Python specific}
\label{deps:python-specific}\begin{itemize}
\item {} 
Python v2.7

\item {} 
Django v1.4

\item {} 
simplejson (included with Django)

\item {} 
MySQLdb Python MySQL bindings

\end{itemize}


\subsection{Misc}
\label{deps:misc}\begin{itemize}
\item {} 
Apache+mod\_wsgi

\item {} 
Any SMTP Server

\item {} 
MySQL Server

\end{itemize}


\subsection{Detailed Instructions}
\label{deps:detailed-instructions}
Below will outline all the required steps to prepare your system.


\subsection{Install Python}
\label{deps:install-python}
If you're using Windows, this will simply be a case of heading to the main
Python website and downloading the Python 2.7 binary and installing. It is
extremely vital that this particular version of Python is used otherwise,
the later (and some earlier) versions are code-incompatible. Either through
syntax differences or some modules are not available.

If you're on Linux, any debian-based system will have the python2 package in
it's repos. Most distros will have the package required.


\subsection{Install Django}
\label{deps:install-django}
If you're using Windows, head to the Django website and download and unzip
the 1.4 source then follow the below:

\begin{Verbatim}[commandchars=\\\{\}]
cd django1.4
python setup.py install
\end{Verbatim}

Other than taking an inordinate amount of time. This will then install Django
1.4 onto your system.

If you're on Linux, please refer to the available packages in your repos to
find the exact name of the package. However, the steps will include:

\begin{Verbatim}[commandchars=\\\{\}]
for Arch:
sudo pacman -S python-django
Debian-based:
sudo apt-get install python-django
\end{Verbatim}


\subsection{simplejson}
\label{deps:simplejson}
This should come with django, but, for whatever reason you wish to use an
externally sourced version, please go to Google and download the latest
via any links found on there.


\subsection{MySQLdb Python bindings}
\label{deps:mysqldb-python-bindings}
This part is actually a little tricky for Windows. Therefore we have decided
to link to a pre-compiled binary for the MySQLdb Python bindings. It can be
found \href{http://www.lfd.uci.edu/~gohlke/pythonlibs/}{here.}

If you're on Linux, please refer to your distro's repos for what the package
name is, however, some hints are below:

\begin{Verbatim}[commandchars=\\\{\}]
on Arch:
sudo pacman -S mysql-python
Debian-based:
sudo apt-get install mysql-python
\end{Verbatim}

This covers the installation portion of the code-dependencies.

Next is preparing your system for the software which needs to be installed.


\subsection{Apache}
\label{deps:apache}
Windows:

Head to the Apache website and download the Apache 2.2 binary and install.

Linux:

\begin{Verbatim}[commandchars=\\\{\}]
On Arch:
sudo pacman -S apache
Debian based:
sudo apt-get install apache
\end{Verbatim}


\subsection{mod\_wsgi}
\label{deps:mod-wsgi}
Download the latest mod\_wsgi.so file from the mod\_wsgi downloads page found,
\href{http://code.google.com/p/modwsgi/wiki/DownloadTheSoftware}{on here.}

Then, depending on the system you are using, put it into your Apache modules
directory.

Windows:

\begin{Verbatim}[commandchars=\\\{\}]
C:\PYGZbs{}Program Files\PYGZbs{}Apache Software Foundation\PYGZbs{}Apache2.2\PYGZbs{}bin\PYGZbs{}
\end{Verbatim}

Linux:

\begin{Verbatim}[commandchars=\\\{\}]
/etc/httpd/modules
\end{Verbatim}

Then for both, modify your httpd.conf file so that it enabled the mod\_wsgi.so
module:

\begin{Verbatim}[commandchars=\\\{\}]
LoadModule wsgi\_module \textless{}path\_to\_modules\_dir\textgreater{}/mod\_wsgi.so
\end{Verbatim}


\section{Code}
\label{code:code}\label{code::doc}

\subsection{Basic Server Settings module}
\label{code:basic-server-settings-module}\label{code:module-timetracker.settings}\index{timetracker.settings (module)}
This module is used for the DJANGO\_SETTINGS\_MODULE when the development server
is being used. This is because there are different levels of logging and the
location of the logs is completely different from when the production server
is being used.

Another reason is that usually (and you should be) the production server is
being run as a daemon user (http user in the case of Apache) and this causes
problems for certain portions of code, particularly logging or anything that
requires write access to the filesystem, thus, the separation of production
settings and the development one.

TODO: Create a base settings module and import attributes from there,
overriding when we need to.


\subsection{Apache Settings module}
\label{code:module-timetracker.apache_settings}\label{code:apache-settings-module}\index{timetracker.apache\_settings (module)}
This module is used for the DJANGO\_SETTINGS\_MODULE for when apache is being
used. This is because there are different levels of logging when using the dev
server and when using the apache server. They also are running under different
user profiles and therefore have different filesystem access rights. On
Windows this isn't a problem but on Linux it makes it a lot easier to use two
separate settings files.

TODO: Create a base settings module and import attributes from there,
overriding when we need to.
\begin{quote}
\begin{quote}\begin{description}
\item[{platform}] \leavevmode
All

\item[{synopsis}] \leavevmode
Module which contains settings for running the app on Apache

\end{description}\end{quote}
\end{quote}


\subsection{timetracker.views}
\label{code:module-timetracker.views}\label{code:timetracker-views}\index{timetracker.views (module)}
Views which are mapped from the URL objects in urls.py
\begin{quote}\begin{description}
\item[{platform}] \leavevmode
All

\item[{synopsis}] \leavevmode
Module which contains view functions that are mapped from urls

\end{description}\end{quote}
\index{add\_change\_user() (in module timetracker.views)}

\begin{fulllineitems}
\phantomsection\label{code:timetracker.views.add_change_user}\pysiglinewithargsret{\code{timetracker.views.}\bfcode{add\_change\_user}}{\emph{request}, \emph{**kwargs}}{}
Creates the view for changing/adding users

This is the view which generates the page to add/edit/change/remove users,
the view first gets the user object from the database, then checks it's
user\_type. If it's an administrator, their authorization table entry is
found then used to create a select box and it's HTML markup. Then pushed
to the template. If it's a team leader, their manager's authorization
table is used instead.
\begin{quote}\begin{description}
\item[{Parameters}] \leavevmode
\textbf{request} -- Automatically passed contains a map of the httprequest

\item[{Returns}] \leavevmode
HttpResponse object back to the browser.

\end{description}\end{quote}

\end{fulllineitems}

\index{admin\_view() (in module timetracker.views)}

\begin{fulllineitems}
\phantomsection\label{code:timetracker.views.admin_view}\pysiglinewithargsret{\code{timetracker.views.}\bfcode{admin\_view}}{\emph{request}, \emph{**kwargs}}{}
This view checks to see if the user logged in is either a team leader or
an administrator. If the user is an administrator, their authorization
table entry is found, iterated over to create a select box and it's HTML
markup, sent to the template. If the user is a team leader, then \emph{their}
manager's authorization table entry is found and used instead. This is to
enable team leaders to view and edit the team in which they are on but
also make it so that we don't explicitly have to duplicate the
authorization table linking the team leader with their team.
\begin{quote}\begin{description}
\item[{Parameters}] \leavevmode
\textbf{request} -- Automatically passed contains a map of the httprequest

\item[{Returns}] \leavevmode
HttpResponse object back to the browser.

\end{description}\end{quote}

\end{fulllineitems}

\index{ajax() (in module timetracker.views)}

\begin{fulllineitems}
\phantomsection\label{code:timetracker.views.ajax}\pysiglinewithargsret{\code{timetracker.views.}\bfcode{ajax}}{\emph{request}}{}
Ajax request handler, dispatches to specific ajax functions depending
on what json gets sent.

Any additional ajax views should be added to the ajax\_funcs map, this will
allow the dispatch function to be used. Future revisions could have a kind
of decorator which could be applied to functions to mutate some global map
of ajax dispatch functions. For now, however, just add them into the map.

The idea for this is that on the client-side call you would construct your
javascript call with something like the below (using jQuery):
\begin{quote}

\begin{Verbatim}[commandchars=\\\{\}]
 \PYG{n+nx}{\PYGZdl{}}\PYG{p}{.}\PYG{n+nx}{ajaxSetup}\PYG{p}{(}\PYG{p}{\PYGZob{}}
     \PYG{n+nx}{type}\PYG{o}{:} \PYG{l+s+s1}{'POST'}\PYG{p}{,}
     \PYG{n+nx}{url}\PYG{o}{:} \PYG{l+s+s1}{'/ajax/'}\PYG{p}{,}
     \PYG{n+nx}{dataType}\PYG{o}{:} \PYG{l+s+s1}{'json'}
 \PYG{p}{\PYGZcb{}}\PYG{p}{)}\PYG{p}{;}

 \PYG{n+nx}{\PYGZdl{}}\PYG{p}{.}\PYG{n+nx}{ajax}\PYG{p}{(}\PYG{p}{\PYGZob{}}
     \PYG{n+nx}{data}\PYG{o}{:} \PYG{p}{\PYGZob{}}
         \PYG{n+nx}{form}\PYG{o}{:} \PYG{l+s+s1}{'functionName'}\PYG{p}{,}
         \PYG{n+nx}{data}\PYG{o}{:} \PYG{l+s+s1}{'data'}
     \PYG{p}{\PYGZcb{}}
\PYG{p}{\PYGZcb{}}\PYG{p}{)}\PYG{p}{;}
\end{Verbatim}
\end{quote}

Using this method, this allows us to construct a single view url and have
all ajax requests come through here. This is highly advantagious because
then we don't have to create a url map and construct views to handle that
specific call. We just have some server-side map and route through there.

The lookup and dispatch works like this:
\begin{enumerate}
\item {} 
Request comes through.

\item {} 
Request gets sent to the ajax view due to the client-side call making a
request to the url mapped to this view.

\item {} 
The form type is detected in the json data sent along with the call.

\item {} 
This string is then pulled out of the dict, executed and it's response
sent back to the browser.

\end{enumerate}
\begin{quote}\begin{description}
\item[{Parameters}] \leavevmode
\textbf{request} -- Automatically passed contains a map of the httprequest

\item[{Returns}] \leavevmode
HttpResponse object back to the browser.

\end{description}\end{quote}

\end{fulllineitems}

\index{edit\_profile() (in module timetracker.views)}

\begin{fulllineitems}
\phantomsection\label{code:timetracker.views.edit_profile}\pysiglinewithargsret{\code{timetracker.views.}\bfcode{edit\_profile}}{\emph{request}, \emph{*args}, \emph{**kwargs}}{}
View for sending the user to the edit profile page

This view is a simple set of fields which allow all kinds of users to edit
pieces of information about their profile, currently it allows uers to
edit their name and their password.
\begin{quote}\begin{description}
\item[{Parameters}] \leavevmode
\textbf{request} -- Automatically passed contains a map of the httprequest

\item[{Returns}] \leavevmode
HttpResponse object back to the browser.

\end{description}\end{quote}

\end{fulllineitems}

\index{explain() (in module timetracker.views)}

\begin{fulllineitems}
\phantomsection\label{code:timetracker.views.explain}\pysiglinewithargsret{\code{timetracker.views.}\bfcode{explain}}{\emph{request}, \emph{*args}, \emph{**kwargs}}{}
Renders the Balance explanation page

This page renders a simple template to show the users how their balance is
calculated. This view takes the user object, retrieves a couple of fields,
which are user.shiftlength and the associated values with that datetime
objects, constructs a string with them and passes it to the template as
the users `shiftlength' attribute. It then takes the count of working
days in the database so that the user has an idea of how many days they
have tracked altogether. Then it calculates their total balance and pushes
all these strings into the template.
\begin{quote}\begin{description}
\item[{Parameters}] \leavevmode
\textbf{request} -- Automatically passed contains a map of the httprequest

\item[{Returns}] \leavevmode
HttpResponse object back to the browser.

\end{description}\end{quote}

\end{fulllineitems}

\index{forgot\_pass() (in module timetracker.views)}

\begin{fulllineitems}
\phantomsection\label{code:timetracker.views.forgot_pass}\pysiglinewithargsret{\code{timetracker.views.}\bfcode{forgot\_pass}}{\emph{request}}{}
Simple view for resetting a user's password

This view has a dual function. The first function is to simply render the
initial page which has a field and the themed markup. On this page a user
can enter their e-mail address and then click submit to have their
password sent to them.

The second function of this page is to respond to the change password
request. In the html markup of the `forgotpass.html' page you will see
that the intention is to have the page post to the same URL which this
page was rendered from. If the request contains POST information then we
retrieve that user from the database, construct an e-mail based on that
and send their password to them. Finally, we redirect to the login page.
\begin{quote}\begin{description}
\item[{Parameters}] \leavevmode
\textbf{request} -- Automatically passed contains a map of the httprequest

\item[{Returns}] \leavevmode
HttpResponse object back to the browser.

\end{description}\end{quote}

\end{fulllineitems}

\index{holiday\_planning() (in module timetracker.views)}

\begin{fulllineitems}
\phantomsection\label{code:timetracker.views.holiday_planning}\pysiglinewithargsret{\code{timetracker.views.}\bfcode{holiday\_planning}}{\emph{request}, \emph{*args}, \emph{**kwargs}}{}
Generates the full holiday table for all employees under a manager

First we find the user object and find whether or not that user is a team
leader or not. If they are a team leader, which set a boolean flag to show
the template what kind of user is logged in. This is so that the team
leaders are not able to view certain things (e.g. Job Codes).

If the admin/tl tries to access the holiday page before any users have
been assigned to them, then we just throw them back to the main page. This
is doubly ensuring that they can't access what would otherwise be a
completely borked page.
\begin{quote}\begin{description}
\item[{Parameters}] \leavevmode
\textbf{request} -- Automatically passed contains a map of the httprequest

\item[{Returns}] \leavevmode
HttpResponse object back to the browser.

\end{description}\end{quote}

\end{fulllineitems}

\index{index() (in module timetracker.views)}

\begin{fulllineitems}
\phantomsection\label{code:timetracker.views.index}\pysiglinewithargsret{\code{timetracker.views.}\bfcode{index}}{\emph{request}}{}
This function serves the base login page. TODO: Make this view check
to see if the user is already logged in and if so, redirect.

This function shouldn't be directly called, it's invocation is automatic
\begin{quote}\begin{description}
\item[{Parameters}] \leavevmode
\textbf{request} -- Automatically passed. Contains a map of the httprequest

\item[{Returns}] \leavevmode
A HttpResponse object which is then passed to the browser

\end{description}\end{quote}

\end{fulllineitems}

\index{login() (in module timetracker.views)}

\begin{fulllineitems}
\phantomsection\label{code:timetracker.views.login}\pysiglinewithargsret{\code{timetracker.views.}\bfcode{login}}{\emph{request}}{}
This function logs the user in, directly adding the session id to
a database entry. This function is invoked from the url mapped in urls.py.
The url is POSTed to, and should contain two fields, the use\_name and the
pass word field. This is then pulled from the database and matched
against, what the user supplied. If they match, the user is then checked
to see what \emph{kind} of user their are, if they are ADMIN or TEAML they will
be sent to the administrator view. Else they will be sent to the user
page.

This function shouldn't be directly called, it's invocation is automatic
from the url mappings.
\begin{quote}\begin{description}
\item[{Parameters}] \leavevmode
\textbf{request} -- Automatically passed. Contains a map of the httprequest

\item[{Returns}] \leavevmode
A HttpResponse object which is then passed to the browser

\end{description}\end{quote}

\end{fulllineitems}

\index{logout() (in module timetracker.views)}

\begin{fulllineitems}
\phantomsection\label{code:timetracker.views.logout}\pysiglinewithargsret{\code{timetracker.views.}\bfcode{logout}}{\emph{request}}{}
Simple logout function

This function will delete a session id from the session dictionary so that
the user will need to log back in order to access the same pages.
\begin{quote}\begin{description}
\item[{Parameters}] \leavevmode
\textbf{request} -- Automatically passed contains a map of the httprequest

\item[{Returns}] \leavevmode
A HttpResponse object which is passed to the browser.

\end{description}\end{quote}

\end{fulllineitems}

\index{user\_view() (in module timetracker.views)}

\begin{fulllineitems}
\phantomsection\label{code:timetracker.views.user_view}\pysiglinewithargsret{\code{timetracker.views.}\bfcode{user\_view}}{\emph{request}, \emph{*args}, \emph{**kwargs}}{}
Generates a calendar based on the URL it receives.
For example: domain.com/calendar/\{year\}/\{month\}/\{day\},
also takes a day just in case you want to add a particular
view for a day, for example. Currently a day-level is not
in-use.
\begin{quote}\begin{description}
\item[{Note }] \leavevmode
The generated HTML should be pretty printed

\item[{Parameters}] \leavevmode\begin{itemize}
\item {} 
\textbf{request} -- Automatically passed contains a map of the httprequest

\item {} 
\textbf{year} -- The year that the view will be rendered with, default is
the current year.

\item {} 
\textbf{month} -- The month that the view will be rendered with, default is
the current month.

\item {} 
\textbf{day} -- The day that the view will be rendered with, default is
the current day

\end{itemize}

\item[{Returns}] \leavevmode
A HttpResponse object which is passed to the browser.

\end{description}\end{quote}

\end{fulllineitems}



\subsection{timetracker.models}
\label{code:module-timetracker.tracker.models}\label{code:timetracker-models}\index{timetracker.tracker.models (module)}
Definition of the models used in the timetracker app
\begin{quote}\begin{description}
\item[{platform}] \leavevmode
All

\item[{synopsis}] \leavevmode
Module which contains view functions that are mapped from urls

\end{description}\end{quote}
\index{Tblauthorization (class in timetracker.tracker.models)}

\begin{fulllineitems}
\phantomsection\label{code:timetracker.tracker.models.Tblauthorization}\pysiglinewithargsret{\strong{class }\code{timetracker.tracker.models.}\bfcode{Tblauthorization}}{\emph{*args}, \emph{**kwargs}}{}
Links Administrators (managers) with their team.

This table is a many-to-many relationship between Administrators and any
other any user type in TEAML/RUSER.

This table is used to explicitly show which people are in an
Administrator's team. Usually in SQL-land, you would be able to
instanstiate multiple rows of many-to-many relationships, however, due to
the fact that working with these tables in an object orientated fashion is
far simpler adding multiple relationships to the same
{\hyperref[code:timetracker.tracker.models.Tblauthorization]{\code{Tblauthorization}}} object, we re-use the relationship when
creating/adding additional {\hyperref[code:timetracker.tracker.models.Tblauthorization]{\code{Tblauthorization}}} instances.

This means that, if you were to need to add a relationship between an
Administrator and a RUSER, then you would need to make sure that you
retrieve the {\hyperref[code:timetracker.tracker.models.Tblauthorization]{\code{Tblauthorization}}} object \emph{before} and save the new
link using that instance. Failure to do this would mean that areas where
the .get() method is employed would start to throw
Tblauthorization.MultipleObjectsReturned.

In future, and time, I would like to make it so that the .save() method is
overloaded and then we can check if a {\hyperref[code:timetracker.tracker.models.Tblauthorization]{\code{Tblauthorization}}} link
already exists and if so, save to that instead.
\index{display\_users() (timetracker.tracker.models.Tblauthorization method)}

\begin{fulllineitems}
\phantomsection\label{code:timetracker.tracker.models.Tblauthorization.display_users}\pysiglinewithargsret{\bfcode{display\_users}}{}{}
Method which generates the HTML for the admin views

This method is depracated in favour of not actually using the admin
interface to interact with {\hyperref[code:timetracker.tracker.models.Tblauthorization]{\code{Tblauthorization}}} instances too
much. That and, it's not unicode-safe.
\begin{quote}\begin{description}
\item[{Return type}] \leavevmode
\code{string}

\end{description}\end{quote}

\end{fulllineitems}

\index{manager\_view() (timetracker.tracker.models.Tblauthorization method)}

\begin{fulllineitems}
\phantomsection\label{code:timetracker.tracker.models.Tblauthorization.manager_view}\pysiglinewithargsret{\bfcode{manager\_view}}{}{}
Method which negates needing to retrieve the users via the
Tblauthorization.objects.all() method which is, needless to say, a
mouthfull.
\begin{quote}\begin{description}
\item[{Return type}] \leavevmode
\code{QuerySet}

\end{description}\end{quote}

\end{fulllineitems}

\index{teamleader\_view() (timetracker.tracker.models.Tblauthorization method)}

\begin{fulllineitems}
\phantomsection\label{code:timetracker.tracker.models.Tblauthorization.teamleader_view}\pysiglinewithargsret{\bfcode{teamleader\_view}}{}{}
Method which provides a shortcut to retrieving the set of users which
are available to the TeamLeader based upon the rules of what they can
access.
\begin{quote}\begin{description}
\item[{Return type}] \leavevmode
\code{QuerySet}

\end{description}\end{quote}

\end{fulllineitems}


\end{fulllineitems}

\index{Tbluser (class in timetracker.tracker.models)}

\begin{fulllineitems}
\phantomsection\label{code:timetracker.tracker.models.Tbluser}\pysiglinewithargsret{\strong{class }\code{timetracker.tracker.models.}\bfcode{Tbluser}}{\emph{*args}, \emph{**kwargs}}{}
Models the user table and provides the admin interface with the
niceties it needs.

This model is the central pillar to this entire application.

The permissions for a user is determined in the view functions, not in a
table this is a design choice because there is only a minimal set of
things to have permissions \emph{over}, so it would be overkill to take full
advantage of the MVC pattern.

User

The most general and base type of a User is the \emph{RUSER}, which is
shorthand (and what actually gets stored in the database) for Regular
User. A regular user will only be able to access a specific section of the
site.

Team Leader

The second type of User is the \emph{TEAML}, this user has very similar level
access as the administrator type but has only a limited subset of their
access rights. They cannot have a team of their own, but view the team
their manager is assigned. They cannot view and/or change job codes, but
they can create new users with all the \emph{other} information that they
need. They can view/create/add/change holidays of themselves and the users
that are assigned to their manager.

Administrator

The third type of User is the Administrator/\emph{ADMIN}. They have full access to all
functions of the app. They can view/create/change/delete members of their
team. They can view/create/add/change holidays of all members of their
team and themselves. They can create users of any type.
\index{display\_user\_type() (timetracker.tracker.models.Tbluser method)}

\begin{fulllineitems}
\phantomsection\label{code:timetracker.tracker.models.Tbluser.display_user_type}\pysiglinewithargsret{\bfcode{display\_user\_type}}{}{}
Function for displaying the user\_type in admin.
\begin{quote}\begin{description}
\item[{Note }] \leavevmode
This method shouldn't be called directly.

\item[{Return type}] \leavevmode
\code{string}

\end{description}\end{quote}

\end{fulllineitems}

\index{get\_administrator() (timetracker.tracker.models.Tbluser method)}

\begin{fulllineitems}
\phantomsection\label{code:timetracker.tracker.models.Tbluser.get_administrator}\pysiglinewithargsret{\bfcode{get\_administrator}}{}{}
Returns the {\hyperref[code:timetracker.tracker.models.Tbluser]{\code{Tbluser}}} who is this instances Authorization link
\begin{quote}\begin{description}
\item[{Returns}] \leavevmode
A {\hyperref[code:timetracker.tracker.models.Tbluser]{\code{Tbluser}}} instance

\item[{Return type}] \leavevmode
{\hyperref[code:timetracker.tracker.models.Tbluser]{\code{Tbluser}}}

\end{description}\end{quote}

\end{fulllineitems}

\index{get\_holiday\_balance() (timetracker.tracker.models.Tbluser method)}

\begin{fulllineitems}
\phantomsection\label{code:timetracker.tracker.models.Tbluser.get_holiday_balance}\pysiglinewithargsret{\bfcode{get\_holiday\_balance}}{\emph{year}}{}
Calculates the holiday balance for the employee

This method loops over all {\hyperref[code:timetracker.tracker.models.TrackingEntry]{\code{TrackingEntry}}} entries attached to
the user instance which are in the year passed in, taking each entries
day\_type and looking that up in a value map.

Values can be:
\begin{enumerate}
\item {} 
Holiday: Remove a day

\item {} 
Work on Public Holiday: Add two days

\end{enumerate}
\begin{enumerate}
\setcounter{enumi}{1}
\item {} 
Return for working Public Holiday: Remove a day

\end{enumerate}
\begin{quote}\begin{description}
\item[{Parameters}] \leavevmode
\textbf{year} (\code{int}) -- The year in which the holiday balance should be
calculated from

\item[{Return type}] \leavevmode
\code{Integer}

\end{description}\end{quote}

\end{fulllineitems}

\index{get\_total\_balance() (timetracker.tracker.models.Tbluser method)}

\begin{fulllineitems}
\phantomsection\label{code:timetracker.tracker.models.Tbluser.get_total_balance}\pysiglinewithargsret{\bfcode{get\_total\_balance}}{\emph{ret='html'}}{}
Calculates the total balance for the user.

This method iterates through every {\hyperref[code:timetracker.tracker.models.TrackingEntry]{\code{TrackingEntry}}} attached to
this user instance which is a working day, multiplies the user's
shiftlength by the number of days and finds the difference between the
projected working hours and the actual working hours.

The return type of this function is different depending on the
argument supplied.
\begin{quote}\begin{description}
\item[{Note }] \leavevmode
To customize how the CSS class is determined when using the
html mode you will need to change the ranges in the
tracking\_class\_map attribute.

\item[{Parameters}] \leavevmode
\textbf{ret} -- Determines the return type of the function. If it is not
supplied then it defaults to `html'. If it is `int' then
the function will return an integer, finally, if the
string `dbg' is passed then we output all the values used
to calculate and the final value.

\item[{Return type}] \leavevmode
\code{string} or \code{integer}

\end{description}\end{quote}

\end{fulllineitems}

\index{is\_admin() (timetracker.tracker.models.Tbluser method)}

\begin{fulllineitems}
\phantomsection\label{code:timetracker.tracker.models.Tbluser.is_admin}\pysiglinewithargsret{\bfcode{is\_admin}}{}{}
Returns whether or not the user instance is an admin type user. The
two types of `admin'-y user\_types are ADMIN and TEAML.
\begin{quote}\begin{description}
\item[{Return type}] \leavevmode
\code{boolean}

\end{description}\end{quote}

\end{fulllineitems}

\index{name() (timetracker.tracker.models.Tbluser method)}

\begin{fulllineitems}
\phantomsection\label{code:timetracker.tracker.models.Tbluser.name}\pysiglinewithargsret{\bfcode{name}}{}{}
Utility method for returning users full name. This is useful for when
we are pretty printing users and their names. For example in e-mails
and or when we are displaying users on the front-end.
\begin{quote}\begin{description}
\item[{Return type}] \leavevmode
\code{string}

\end{description}\end{quote}

\end{fulllineitems}

\index{tracking\_entries() (timetracker.tracker.models.Tbluser method)}

\begin{fulllineitems}
\phantomsection\label{code:timetracker.tracker.models.Tbluser.tracking_entries}\pysiglinewithargsret{\bfcode{tracking\_entries}}{\emph{year=2012}, \emph{month=5}}{}
Returns all the tracking entries associated with
this user.

This is particularly useful when required to make a report or generate
a specific view of the tracking entries of the user.
\begin{quote}\begin{description}
\item[{Parameters}] \leavevmode\begin{itemize}
\item {} 
\textbf{year} -- The year in which the QuerySet should be filtered
by. Defaults to the current year.

\item {} 
\textbf{month} -- The month in which the QuerySet should be filtered
by. Defaults to the current month.

\end{itemize}

\item[{Return type}] \leavevmode
\code{QuerySet}

\end{description}\end{quote}

\end{fulllineitems}


\end{fulllineitems}

\index{TrackingEntry (class in timetracker.tracker.models)}

\begin{fulllineitems}
\phantomsection\label{code:timetracker.tracker.models.TrackingEntry}\pysiglinewithargsret{\strong{class }\code{timetracker.tracker.models.}\bfcode{TrackingEntry}}{\emph{*args}, \emph{**kwargs}}{}
Model which is used to enter working logs into the database.

A tracking entry consists of several fields:-
\begin{enumerate}
\item {} 
Entry date: The date that the working log happened.

\item {} 
Start Time: The start time of the working day.

\item {} 
End Time: The end time of the working day.

\item {} 
Breaks: Any breaks taken during that day.

\item {} 
Day Type: The type of working log.

\end{enumerate}

Again, the TrackingEntry model is a core component of the time tracking
application. It directly links users with the time-spent at work and the
the type of day that was.

\end{fulllineitems}



\subsection{timetracker.utils.calendar\_utils}
\label{code:timetracker-utils-calendar-utils}\label{code:module-timetracker.utils.calendar_utils}\index{timetracker.utils.calendar\_utils (module)}
Module for collecting the utility functions dealing with mostly calendar
tasks, processing dates and creating time-based code.
\index{ajax\_add\_entry() (in module timetracker.utils.calendar\_utils)}

\begin{fulllineitems}
\phantomsection\label{code:timetracker.utils.calendar_utils.ajax_add_entry}\pysiglinewithargsret{\code{timetracker.utils.calendar\_utils.}\bfcode{ajax\_add\_entry}}{\emph{request}}{}
Adds a calendar entry asynchronously

\end{fulllineitems}

\index{ajax\_change\_entry() (in module timetracker.utils.calendar\_utils)}

\begin{fulllineitems}
\phantomsection\label{code:timetracker.utils.calendar_utils.ajax_change_entry}\pysiglinewithargsret{\code{timetracker.utils.calendar\_utils.}\bfcode{ajax\_change\_entry}}{\emph{request}}{}
Changes a calendar entry asynchronously

\end{fulllineitems}

\index{ajax\_delete\_entry() (in module timetracker.utils.calendar\_utils)}

\begin{fulllineitems}
\phantomsection\label{code:timetracker.utils.calendar_utils.ajax_delete_entry}\pysiglinewithargsret{\code{timetracker.utils.calendar\_utils.}\bfcode{ajax\_delete\_entry}}{\emph{request}}{}
Asynchronously deletes an entry

\end{fulllineitems}

\index{ajax\_error() (in module timetracker.utils.calendar\_utils)}

\begin{fulllineitems}
\phantomsection\label{code:timetracker.utils.calendar_utils.ajax_error}\pysiglinewithargsret{\code{timetracker.utils.calendar\_utils.}\bfcode{ajax\_error}}{\emph{request}}{}
Returns a HttpResponse with JSON as a payload with the error
code as the string that the function is called with

\end{fulllineitems}

\index{calendar\_wrapper() (in module timetracker.utils.calendar\_utils)}

\begin{fulllineitems}
\phantomsection\label{code:timetracker.utils.calendar_utils.calendar_wrapper}\pysiglinewithargsret{\code{timetracker.utils.calendar\_utils.}\bfcode{calendar\_wrapper}}{\emph{function}}{}
Decorator which checks if the calendar function was
called as an ajax request or not, if so, then the
the wrapper constructs the arguments for the call
from the POST items
\begin{quote}\begin{description}
\item[{Parameters}] \leavevmode
\textbf{function} -- Literally just gen\_calendar.

\item[{Return type}] \leavevmode
Nothing directly because it returns gen\_calendar's

\end{description}\end{quote}

\end{fulllineitems}

\index{delete\_user() (in module timetracker.utils.calendar\_utils)}

\begin{fulllineitems}
\phantomsection\label{code:timetracker.utils.calendar_utils.delete_user}\pysiglinewithargsret{\code{timetracker.utils.calendar\_utils.}\bfcode{delete\_user}}{\emph{request}}{}
Asynchronously deletes a user

\end{fulllineitems}

\index{gen\_calendar() (in module timetracker.utils.calendar\_utils)}

\begin{fulllineitems}
\phantomsection\label{code:timetracker.utils.calendar_utils.gen_calendar}\pysiglinewithargsret{\code{timetracker.utils.calendar\_utils.}\bfcode{gen\_calendar}}{\emph{*args}, \emph{**kwargs}}{}
Returns a HTML calendar, calling a database user to get their day-by-day
entries and gives each day a special CSS class so that days can be styled
individually.

The generated HTML should be `pretty printed' as well, so the output code
should be pretty readable.

\end{fulllineitems}

\index{gen\_datetime\_cal() (in module timetracker.utils.calendar\_utils)}

\begin{fulllineitems}
\phantomsection\label{code:timetracker.utils.calendar_utils.gen_datetime_cal}\pysiglinewithargsret{\code{timetracker.utils.calendar\_utils.}\bfcode{gen\_datetime\_cal}}{\emph{year}, \emph{month}}{}
Generates a datetime list of all days in a month

\end{fulllineitems}

\index{gen\_holiday\_list() (in module timetracker.utils.calendar\_utils)}

\begin{fulllineitems}
\phantomsection\label{code:timetracker.utils.calendar_utils.gen_holiday_list}\pysiglinewithargsret{\code{timetracker.utils.calendar\_utils.}\bfcode{gen\_holiday\_list}}{\emph{admin\_user}, \emph{year=2012}, \emph{month=5}}{}
Outputs a holiday calendar for that month.

For each user we get their tracking entries, then iterate over each of
their entries checking if it is a holiday or not, if it is then we change
the class entry for that number in the day class' dict. Adds a submit
button along with passing the user\_id to it.
\begin{quote}\begin{description}
\item[{Parameters}] \leavevmode\begin{itemize}
\item {} 
\textbf{admin\_user} -- {\hyperref[code:timetracker.tracker.models.Tbluser]{\code{timetracker.tracker.models.Tbluser}}} instance.

\item {} 
\textbf{year} -- \code{int} of the year required to be output, defaults to
the current year.

\item {} 
\textbf{month} -- \code{int} of the month required to be output, defaults to
the current month.

\end{itemize}

\item[{Returns}] \leavevmode
A partially pretty printed html string.

\item[{Return type}] \leavevmode
\code{str}

\end{description}\end{quote}

\end{fulllineitems}

\index{get\_request\_data() (in module timetracker.utils.calendar\_utils)}

\begin{fulllineitems}
\phantomsection\label{code:timetracker.utils.calendar_utils.get_request_data}\pysiglinewithargsret{\code{timetracker.utils.calendar\_utils.}\bfcode{get\_request\_data}}{\emph{form}, \emph{request}}{}
Given a form and a request object we pull out
from the request what the form defines.

i.e.:

\begin{Verbatim}[commandchars=\\\{\}]
\PYG{n}{form} \PYG{o}{=} \PYG{p}{\PYGZob{}}
     \PYG{l+s}{'}\PYG{l+s}{data1}\PYG{l+s}{'}\PYG{p}{:} \PYG{n+nb+bp}{None}
\PYG{p}{\PYGZcb{}}
\end{Verbatim}

get\_request\_data(form, request) will then fill that data with what's in
the request object.
\begin{quote}\begin{description}
\item[{Parameters}] \leavevmode\begin{itemize}
\item {} 
\textbf{form} -- A dictionary of items which should be filled from

\item {} 
\textbf{request} -- The request object where the data should be taken from.

\end{itemize}

\item[{Returns}] \leavevmode
A dictionary which contains the actual data from the request.

\item[{Return type}] \leavevmode
\code{dict}

\end{description}\end{quote}

\end{fulllineitems}

\index{get\_user\_data() (in module timetracker.utils.calendar\_utils)}

\begin{fulllineitems}
\phantomsection\label{code:timetracker.utils.calendar_utils.get_user_data}\pysiglinewithargsret{\code{timetracker.utils.calendar\_utils.}\bfcode{get\_user\_data}}{\emph{request}}{}
Returns a user as a json object

\end{fulllineitems}

\index{mass\_holidays() (in module timetracker.utils.calendar\_utils)}

\begin{fulllineitems}
\phantomsection\label{code:timetracker.utils.calendar_utils.mass_holidays}\pysiglinewithargsret{\code{timetracker.utils.calendar\_utils.}\bfcode{mass\_holidays}}{\emph{request}}{}
Adds a holidays for a specific user en masse

\end{fulllineitems}

\index{parse\_time() (in module timetracker.utils.calendar\_utils)}

\begin{fulllineitems}
\phantomsection\label{code:timetracker.utils.calendar_utils.parse_time}\pysiglinewithargsret{\code{timetracker.utils.calendar\_utils.}\bfcode{parse\_time}}{\emph{timestring}, \emph{type\_of=\textless{}type `int'\textgreater{}}}{}
Given a time string will return a tuple of ints,
i.e. ``09:44'' returns {[}9, 44{]} with the default args,
you can pass any function to the type argument.
\begin{quote}\begin{description}
\item[{Parameters}] \leavevmode\begin{itemize}
\item {} 
\textbf{timestring} -- String such as `09:44'

\item {} 
\textbf{type\_of} -- A type which the split string should be converted to,
suitable types are: \code{int}, \code{str} and
\code{float}.

\end{itemize}

\end{description}\end{quote}

\end{fulllineitems}

\index{profile\_edit() (in module timetracker.utils.calendar\_utils)}

\begin{fulllineitems}
\phantomsection\label{code:timetracker.utils.calendar_utils.profile_edit}\pysiglinewithargsret{\code{timetracker.utils.calendar\_utils.}\bfcode{profile\_edit}}{\emph{request}}{}
Asynchronously edits a user's profile

\end{fulllineitems}

\index{useredit() (in module timetracker.utils.calendar\_utils)}

\begin{fulllineitems}
\phantomsection\label{code:timetracker.utils.calendar_utils.useredit}\pysiglinewithargsret{\code{timetracker.utils.calendar\_utils.}\bfcode{useredit}}{\emph{request}}{}
Adds a user to the database asynchronously

\end{fulllineitems}

\index{validate\_time() (in module timetracker.utils.calendar\_utils)}

\begin{fulllineitems}
\phantomsection\label{code:timetracker.utils.calendar_utils.validate_time}\pysiglinewithargsret{\code{timetracker.utils.calendar\_utils.}\bfcode{validate\_time}}{\emph{start}, \emph{end}}{}
Validates that the start time is before the end time
\begin{quote}\begin{description}
\item[{Parameters}] \leavevmode\begin{itemize}
\item {} 
\textbf{start} -- String time such as ``09:45''

\item {} 
\textbf{end} -- String time such as ``17:00''

\end{itemize}

\item[{Return type}] \leavevmode
\code{boolean}

\end{description}\end{quote}

\end{fulllineitems}



\chapter{Indices and tables}
\label{index:indices-and-tables}\begin{itemize}
\item {} 
\emph{genindex}

\item {} 
\emph{modindex}

\item {} 
\emph{search}

\end{itemize}


\renewcommand{\indexname}{Python Module Index}
\begin{theindex}
\def\bigletter#1{{\Large\sffamily#1}\nopagebreak\vspace{1mm}}
\bigletter{t}
\item {\texttt{timetracker.apache\_settings}}, \pageref{code:module-timetracker.apache_settings}
\item {\texttt{timetracker.settings}}, \pageref{code:module-timetracker.settings}
\item {\texttt{timetracker.tracker.models}}, \pageref{code:module-timetracker.tracker.models}
\item {\texttt{timetracker.utils.calendar\_utils}}, \pageref{code:module-timetracker.utils.calendar_utils}
\item {\texttt{timetracker.views}}, \pageref{code:module-timetracker.views}
\end{theindex}

\renewcommand{\indexname}{Index}
\printindex
\end{document}
