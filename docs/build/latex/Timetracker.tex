% Generated by Sphinx.
\def\sphinxdocclass{report}
\documentclass[letterpaper,10pt,english]{sphinxmanual}
\usepackage[utf8]{inputenc}
\DeclareUnicodeCharacter{00A0}{\nobreakspace}
\usepackage[T1]{fontenc}
\usepackage{babel}
\usepackage{times}
\usepackage[Bjarne]{fncychap}
\usepackage{longtable}
\usepackage{sphinx}
\usepackage{multirow}


\title{Timetracker Documentation}
\date{May 23, 2012}
\release{1}
\author{Aaron France}
\newcommand{\sphinxlogo}{}
\renewcommand{\releasename}{Release}
\makeindex

\makeatletter
\def\PYG@reset{\let\PYG@it=\relax \let\PYG@bf=\relax%
    \let\PYG@ul=\relax \let\PYG@tc=\relax%
    \let\PYG@bc=\relax \let\PYG@ff=\relax}
\def\PYG@tok#1{\csname PYG@tok@#1\endcsname}
\def\PYG@toks#1+{\ifx\relax#1\empty\else%
    \PYG@tok{#1}\expandafter\PYG@toks\fi}
\def\PYG@do#1{\PYG@bc{\PYG@tc{\PYG@ul{%
    \PYG@it{\PYG@bf{\PYG@ff{#1}}}}}}}
\def\PYG#1#2{\PYG@reset\PYG@toks#1+\relax+\PYG@do{#2}}

\expandafter\def\csname PYG@tok@gd\endcsname{\def\PYG@tc##1{\textcolor[rgb]{0.63,0.00,0.00}{##1}}}
\expandafter\def\csname PYG@tok@gu\endcsname{\let\PYG@bf=\textbf\def\PYG@tc##1{\textcolor[rgb]{0.50,0.00,0.50}{##1}}}
\expandafter\def\csname PYG@tok@gt\endcsname{\def\PYG@tc##1{\textcolor[rgb]{0.00,0.25,0.82}{##1}}}
\expandafter\def\csname PYG@tok@gs\endcsname{\let\PYG@bf=\textbf}
\expandafter\def\csname PYG@tok@gr\endcsname{\def\PYG@tc##1{\textcolor[rgb]{1.00,0.00,0.00}{##1}}}
\expandafter\def\csname PYG@tok@cm\endcsname{\let\PYG@it=\textit\def\PYG@tc##1{\textcolor[rgb]{0.25,0.50,0.56}{##1}}}
\expandafter\def\csname PYG@tok@vg\endcsname{\def\PYG@tc##1{\textcolor[rgb]{0.73,0.38,0.84}{##1}}}
\expandafter\def\csname PYG@tok@m\endcsname{\def\PYG@tc##1{\textcolor[rgb]{0.13,0.50,0.31}{##1}}}
\expandafter\def\csname PYG@tok@mh\endcsname{\def\PYG@tc##1{\textcolor[rgb]{0.13,0.50,0.31}{##1}}}
\expandafter\def\csname PYG@tok@cs\endcsname{\def\PYG@tc##1{\textcolor[rgb]{0.25,0.50,0.56}{##1}}\def\PYG@bc##1{\setlength{\fboxsep}{0pt}\colorbox[rgb]{1.00,0.94,0.94}{\strut ##1}}}
\expandafter\def\csname PYG@tok@ge\endcsname{\let\PYG@it=\textit}
\expandafter\def\csname PYG@tok@vc\endcsname{\def\PYG@tc##1{\textcolor[rgb]{0.73,0.38,0.84}{##1}}}
\expandafter\def\csname PYG@tok@il\endcsname{\def\PYG@tc##1{\textcolor[rgb]{0.13,0.50,0.31}{##1}}}
\expandafter\def\csname PYG@tok@go\endcsname{\def\PYG@tc##1{\textcolor[rgb]{0.19,0.19,0.19}{##1}}}
\expandafter\def\csname PYG@tok@cp\endcsname{\def\PYG@tc##1{\textcolor[rgb]{0.00,0.44,0.13}{##1}}}
\expandafter\def\csname PYG@tok@gi\endcsname{\def\PYG@tc##1{\textcolor[rgb]{0.00,0.63,0.00}{##1}}}
\expandafter\def\csname PYG@tok@gh\endcsname{\let\PYG@bf=\textbf\def\PYG@tc##1{\textcolor[rgb]{0.00,0.00,0.50}{##1}}}
\expandafter\def\csname PYG@tok@ni\endcsname{\let\PYG@bf=\textbf\def\PYG@tc##1{\textcolor[rgb]{0.84,0.33,0.22}{##1}}}
\expandafter\def\csname PYG@tok@nl\endcsname{\let\PYG@bf=\textbf\def\PYG@tc##1{\textcolor[rgb]{0.00,0.13,0.44}{##1}}}
\expandafter\def\csname PYG@tok@nn\endcsname{\let\PYG@bf=\textbf\def\PYG@tc##1{\textcolor[rgb]{0.05,0.52,0.71}{##1}}}
\expandafter\def\csname PYG@tok@no\endcsname{\def\PYG@tc##1{\textcolor[rgb]{0.38,0.68,0.84}{##1}}}
\expandafter\def\csname PYG@tok@na\endcsname{\def\PYG@tc##1{\textcolor[rgb]{0.25,0.44,0.63}{##1}}}
\expandafter\def\csname PYG@tok@nb\endcsname{\def\PYG@tc##1{\textcolor[rgb]{0.00,0.44,0.13}{##1}}}
\expandafter\def\csname PYG@tok@nc\endcsname{\let\PYG@bf=\textbf\def\PYG@tc##1{\textcolor[rgb]{0.05,0.52,0.71}{##1}}}
\expandafter\def\csname PYG@tok@nd\endcsname{\let\PYG@bf=\textbf\def\PYG@tc##1{\textcolor[rgb]{0.33,0.33,0.33}{##1}}}
\expandafter\def\csname PYG@tok@ne\endcsname{\def\PYG@tc##1{\textcolor[rgb]{0.00,0.44,0.13}{##1}}}
\expandafter\def\csname PYG@tok@nf\endcsname{\def\PYG@tc##1{\textcolor[rgb]{0.02,0.16,0.49}{##1}}}
\expandafter\def\csname PYG@tok@si\endcsname{\let\PYG@it=\textit\def\PYG@tc##1{\textcolor[rgb]{0.44,0.63,0.82}{##1}}}
\expandafter\def\csname PYG@tok@s2\endcsname{\def\PYG@tc##1{\textcolor[rgb]{0.25,0.44,0.63}{##1}}}
\expandafter\def\csname PYG@tok@vi\endcsname{\def\PYG@tc##1{\textcolor[rgb]{0.73,0.38,0.84}{##1}}}
\expandafter\def\csname PYG@tok@nt\endcsname{\let\PYG@bf=\textbf\def\PYG@tc##1{\textcolor[rgb]{0.02,0.16,0.45}{##1}}}
\expandafter\def\csname PYG@tok@nv\endcsname{\def\PYG@tc##1{\textcolor[rgb]{0.73,0.38,0.84}{##1}}}
\expandafter\def\csname PYG@tok@s1\endcsname{\def\PYG@tc##1{\textcolor[rgb]{0.25,0.44,0.63}{##1}}}
\expandafter\def\csname PYG@tok@gp\endcsname{\let\PYG@bf=\textbf\def\PYG@tc##1{\textcolor[rgb]{0.78,0.36,0.04}{##1}}}
\expandafter\def\csname PYG@tok@sh\endcsname{\def\PYG@tc##1{\textcolor[rgb]{0.25,0.44,0.63}{##1}}}
\expandafter\def\csname PYG@tok@ow\endcsname{\let\PYG@bf=\textbf\def\PYG@tc##1{\textcolor[rgb]{0.00,0.44,0.13}{##1}}}
\expandafter\def\csname PYG@tok@sx\endcsname{\def\PYG@tc##1{\textcolor[rgb]{0.78,0.36,0.04}{##1}}}
\expandafter\def\csname PYG@tok@bp\endcsname{\def\PYG@tc##1{\textcolor[rgb]{0.00,0.44,0.13}{##1}}}
\expandafter\def\csname PYG@tok@c1\endcsname{\let\PYG@it=\textit\def\PYG@tc##1{\textcolor[rgb]{0.25,0.50,0.56}{##1}}}
\expandafter\def\csname PYG@tok@kc\endcsname{\let\PYG@bf=\textbf\def\PYG@tc##1{\textcolor[rgb]{0.00,0.44,0.13}{##1}}}
\expandafter\def\csname PYG@tok@c\endcsname{\let\PYG@it=\textit\def\PYG@tc##1{\textcolor[rgb]{0.25,0.50,0.56}{##1}}}
\expandafter\def\csname PYG@tok@mf\endcsname{\def\PYG@tc##1{\textcolor[rgb]{0.13,0.50,0.31}{##1}}}
\expandafter\def\csname PYG@tok@err\endcsname{\def\PYG@bc##1{\setlength{\fboxsep}{0pt}\fcolorbox[rgb]{1.00,0.00,0.00}{1,1,1}{\strut ##1}}}
\expandafter\def\csname PYG@tok@kd\endcsname{\let\PYG@bf=\textbf\def\PYG@tc##1{\textcolor[rgb]{0.00,0.44,0.13}{##1}}}
\expandafter\def\csname PYG@tok@ss\endcsname{\def\PYG@tc##1{\textcolor[rgb]{0.32,0.47,0.09}{##1}}}
\expandafter\def\csname PYG@tok@sr\endcsname{\def\PYG@tc##1{\textcolor[rgb]{0.14,0.33,0.53}{##1}}}
\expandafter\def\csname PYG@tok@mo\endcsname{\def\PYG@tc##1{\textcolor[rgb]{0.13,0.50,0.31}{##1}}}
\expandafter\def\csname PYG@tok@mi\endcsname{\def\PYG@tc##1{\textcolor[rgb]{0.13,0.50,0.31}{##1}}}
\expandafter\def\csname PYG@tok@kn\endcsname{\let\PYG@bf=\textbf\def\PYG@tc##1{\textcolor[rgb]{0.00,0.44,0.13}{##1}}}
\expandafter\def\csname PYG@tok@o\endcsname{\def\PYG@tc##1{\textcolor[rgb]{0.40,0.40,0.40}{##1}}}
\expandafter\def\csname PYG@tok@kr\endcsname{\let\PYG@bf=\textbf\def\PYG@tc##1{\textcolor[rgb]{0.00,0.44,0.13}{##1}}}
\expandafter\def\csname PYG@tok@s\endcsname{\def\PYG@tc##1{\textcolor[rgb]{0.25,0.44,0.63}{##1}}}
\expandafter\def\csname PYG@tok@kp\endcsname{\def\PYG@tc##1{\textcolor[rgb]{0.00,0.44,0.13}{##1}}}
\expandafter\def\csname PYG@tok@w\endcsname{\def\PYG@tc##1{\textcolor[rgb]{0.73,0.73,0.73}{##1}}}
\expandafter\def\csname PYG@tok@kt\endcsname{\def\PYG@tc##1{\textcolor[rgb]{0.56,0.13,0.00}{##1}}}
\expandafter\def\csname PYG@tok@sc\endcsname{\def\PYG@tc##1{\textcolor[rgb]{0.25,0.44,0.63}{##1}}}
\expandafter\def\csname PYG@tok@sb\endcsname{\def\PYG@tc##1{\textcolor[rgb]{0.25,0.44,0.63}{##1}}}
\expandafter\def\csname PYG@tok@k\endcsname{\let\PYG@bf=\textbf\def\PYG@tc##1{\textcolor[rgb]{0.00,0.44,0.13}{##1}}}
\expandafter\def\csname PYG@tok@se\endcsname{\let\PYG@bf=\textbf\def\PYG@tc##1{\textcolor[rgb]{0.25,0.44,0.63}{##1}}}
\expandafter\def\csname PYG@tok@sd\endcsname{\let\PYG@it=\textit\def\PYG@tc##1{\textcolor[rgb]{0.25,0.44,0.63}{##1}}}

\def\PYGZbs{\char`\\}
\def\PYGZus{\char`\_}
\def\PYGZob{\char`\{}
\def\PYGZcb{\char`\}}
\def\PYGZca{\char`\^}
\def\PYGZam{\char`\&}
\def\PYGZlt{\char`\<}
\def\PYGZgt{\char`\>}
\def\PYGZsh{\char`\#}
\def\PYGZpc{\char`\%}
\def\PYGZdl{\char`\$}
\def\PYGZti{\char`\~}
% for compatibility with earlier versions
\def\PYGZat{@}
\def\PYGZlb{[}
\def\PYGZrb{]}
\makeatother

\begin{document}

\maketitle
\tableofcontents
\phantomsection\label{index::doc}



\chapter{Contents:}
\label{index:contents}\label{index:welcome-to-timetracker-s-documentation}

\section{Dependencies}
\label{deps:dependencies}\label{deps::doc}

\subsection{Python specific}
\label{deps:python-specific}\begin{itemize}
\item {} 
Python v2.7

\item {} 
Django v1.4

\item {} 
simplejson (included with Django)

\item {} 
MySQLdb Python MySQL bindings

\end{itemize}


\subsection{Misc}
\label{deps:misc}\begin{itemize}
\item {} 
Apache+mod\_wsgi

\item {} 
Any SMTP Server

\item {} 
MySQL Server

\end{itemize}


\subsection{Detailed Instructions}
\label{deps:detailed-instructions}
Below will outline all the required steps to prepare your system.


\subsection{Install Python}
\label{deps:install-python}
If you're using Windows, this will simply be a case of heading to the main
Python website and downloading the Python 2.7 binary and installing. It is
extremely vital that this particular version of Python is used otherwise,
the later (and some earlier) versions are code-incompatible. Either through
syntax differences or some modules are not available.

If you're on Linux, any debian-based system will have the python2 package in
it's repos. Most distros will have the package required.


\subsection{Install Django}
\label{deps:install-django}
If you're using Windows, head to the Django website and download and unzip
the 1.4 source then follow the below:

\begin{Verbatim}[commandchars=\\\{\}]
cd django1.4
python setup.py install
\end{Verbatim}

Other than taking an inordinate amount of time. This will then install Django
1.4 onto your system.

If you're on Linux, please refer to the available packages in your repos to
find the exact name of the package. However, the steps will include:

\begin{Verbatim}[commandchars=\\\{\}]
for Arch:
sudo pacman -S python-django
Debian-based:
sudo apt-get install python-django
\end{Verbatim}


\subsection{simplejson}
\label{deps:simplejson}
This should come with django, but, for whatever reason you wish to use an
externally sourced version, please go to Google and download the latest
via any links found on there.


\subsection{MySQLdb Python bindings}
\label{deps:mysqldb-python-bindings}
This part is actually a little tricky for Windows. Therefore we have decided
to link to a pre-compiled binary for the MySQLdb Python bindings. It can be
found \href{http://www.lfd.uci.edu/~gohlke/pythonlibs/}{here.}

If you're on Linux, please refer to your distro's repos for what the package
name is, however, some hints are below:

\begin{Verbatim}[commandchars=\\\{\}]
on Arch:
sudo pacman -S mysql-python
Debian-based:
sudo apt-get install mysql-python
\end{Verbatim}

This covers the installation portion of the code-dependencies.

Next is preparing your system for the software which needs to be installed.


\subsection{Apache}
\label{deps:apache}
Windows:

Head to the Apache website and download the Apache 2.2 binary and install.

Linux:

\begin{Verbatim}[commandchars=\\\{\}]
On Arch:
sudo pacman -S apache
Debian based:
sudo apt-get install apache
\end{Verbatim}


\subsection{mod\_wsgi}
\label{deps:mod-wsgi}
Download the latest mod\_wsgi.so file from the mod\_wsgi downloads page found,
\href{http://code.google.com/p/modwsgi/wiki/DownloadTheSoftware}{on here.}

Then, depending on the system you are using, put it into your Apache modules
directory.

Windows:

\begin{Verbatim}[commandchars=\\\{\}]
C:\PYGZbs{}Program Files\PYGZbs{}Apache Software Foundation\PYGZbs{}Apache2.2\PYGZbs{}bin\PYGZbs{}
\end{Verbatim}

Linux:

\begin{Verbatim}[commandchars=\\\{\}]
/etc/httpd/modules
\end{Verbatim}

Then for both, modify your httpd.conf file so that it enabled the mod\_wsgi.so
module:

\begin{Verbatim}[commandchars=\\\{\}]
LoadModule wsgi\_module \textless{}path\_to\_modules\_dir\textgreater{}/mod\_wsgi.so
\end{Verbatim}


\section{Settings files}
\label{settings:settings-files}\label{settings::doc}

\subsection{Basic Server Settings module}
\label{settings:basic-server-settings-module}\label{settings:module-timetracker.settings}\index{timetracker.settings (module)}
This module is used for the DJANGO\_SETTINGS\_MODULE when the development server
is being used. This is because there are different levels of logging and the
location of the logs is completely different from when the production server
is being used.

Another reason is that usually (and you should be) the production server is
being run as a daemon user (http user in the case of Apache) and this causes
problems for certain portions of code, particularly logging or anything that
requires write access to the filesystem, thus, the separation of production
settings and the development one.

TODO: Create a base settings module and import attributes from there,
overriding when we need to.


\subsection{Apache Settings module}
\label{settings:module-timetracker.apache_settings}\label{settings:apache-settings-module}\index{timetracker.apache\_settings (module)}
This module is used for the DJANGO\_SETTINGS\_MODULE for when apache is being
used. This is because there are different levels of logging when using the dev
server and when using the apache server. They also are running under different
user profiles and therefore have different filesystem access rights. On
Windows this isn't a problem but on Linux it makes it a lot easier to use two
separate settings files.

TODO: Create a base settings module and import attributes from there,
overriding when we need to.
\begin{quote}
\begin{quote}\begin{description}
\item[{platform}] \leavevmode
All

\item[{synopsis}] \leavevmode
Module which contains settings for running the app on Apache

\end{description}\end{quote}
\end{quote}


\section{Timetracker API and code}
\label{timetracker:timetracker-api-and-code}\label{timetracker::doc}
Here I will outline the API for the timetracker application and any advice I
have pertaining to the inner-workings of the code itself.


\subsection{timetracker.views}
\label{timetracker:module-timetracker.views}\label{timetracker:timetracker-views}\index{timetracker.views (module)}
Views which are mapped from the URL objects in urls.py
\begin{quote}\begin{description}
\item[{platform}] \leavevmode
All

\item[{synopsis}] \leavevmode
Module which contains view functions that are mapped from urls

\end{description}\end{quote}
\index{add\_change\_user() (in module timetracker.views)}

\begin{fulllineitems}
\phantomsection\label{timetracker:timetracker.views.add_change_user}\pysiglinewithargsret{\code{timetracker.views.}\bfcode{add\_change\_user}}{\emph{request}, \emph{**kwargs}}{}
Creates the view for changing/adding users

This is the view which generates the page to add/edit/change/remove users,
the view first gets the user object from the database, then checks it's
user\_type. If it's an administrator, their authorization table entry is
found then used to create a select box and it's HTML markup. Then pushed
to the template. If it's a team leader, their manager's authorization
table is used instead.
\begin{quote}\begin{description}
\item[{Parameters}] \leavevmode
\textbf{request} -- Automatically passed contains a map of the httprequest

\item[{Returns}] \leavevmode
HttpResponse object back to the browser.

\end{description}\end{quote}

\end{fulllineitems}

\index{admin\_view() (in module timetracker.views)}

\begin{fulllineitems}
\phantomsection\label{timetracker:timetracker.views.admin_view}\pysiglinewithargsret{\code{timetracker.views.}\bfcode{admin\_view}}{\emph{request}, \emph{**kwargs}}{}
This view checks to see if the user logged in is either a team leader or
an administrator. If the user is an administrator, their authorization
table entry is found, iterated over to create a select box and it's HTML
markup, sent to the template. If the user is a team leader, then \emph{their}
manager's authorization table entry is found and used instead. This is to
enable team leaders to view and edit the team in which they are on but
also make it so that we don't explicitly have to duplicate the
authorization table linking the team leader with their team.
\begin{quote}\begin{description}
\item[{Parameters}] \leavevmode
\textbf{request} -- Automatically passed contains a map of the httprequest

\item[{Returns}] \leavevmode
HttpResponse object back to the browser.

\end{description}\end{quote}

\end{fulllineitems}

\index{ajax() (in module timetracker.views)}

\begin{fulllineitems}
\phantomsection\label{timetracker:timetracker.views.ajax}\pysiglinewithargsret{\code{timetracker.views.}\bfcode{ajax}}{\emph{request}}{}
Ajax request handler, dispatches to specific ajax functions depending
on what json gets sent.

Any additional ajax views should be added to the ajax\_funcs map, this will
allow the dispatch function to be used. Future revisions could have a kind
of decorator which could be applied to functions to mutate some global map
of ajax dispatch functions. For now, however, just add them into the map.

The idea for this is that on the client-side call you would construct your
javascript call with something like the below (using jQuery):
\begin{quote}

\begin{Verbatim}[commandchars=\\\{\}]
 \PYG{n+nx}{\PYGZdl{}}\PYG{p}{.}\PYG{n+nx}{ajaxSetup}\PYG{p}{(}\PYG{p}{\PYGZob{}}
     \PYG{n+nx}{type}\PYG{o}{:} \PYG{l+s+s1}{'POST'}\PYG{p}{,}
     \PYG{n+nx}{url}\PYG{o}{:} \PYG{l+s+s1}{'/ajax/'}\PYG{p}{,}
     \PYG{n+nx}{dataType}\PYG{o}{:} \PYG{l+s+s1}{'json'}
 \PYG{p}{\PYGZcb{}}\PYG{p}{)}\PYG{p}{;}

 \PYG{n+nx}{\PYGZdl{}}\PYG{p}{.}\PYG{n+nx}{ajax}\PYG{p}{(}\PYG{p}{\PYGZob{}}
     \PYG{n+nx}{data}\PYG{o}{:} \PYG{p}{\PYGZob{}}
         \PYG{n+nx}{form}\PYG{o}{:} \PYG{l+s+s1}{'functionName'}\PYG{p}{,}
         \PYG{n+nx}{data}\PYG{o}{:} \PYG{l+s+s1}{'data'}
     \PYG{p}{\PYGZcb{}}
\PYG{p}{\PYGZcb{}}\PYG{p}{)}\PYG{p}{;}
\end{Verbatim}
\end{quote}

Using this method, this allows us to construct a single view url and have
all ajax requests come through here. This is highly advantagious because
then we don't have to create a url map and construct views to handle that
specific call. We just have some server-side map and route through there.

The lookup and dispatch works like this:
\begin{enumerate}
\item {} 
Request comes through.

\item {} 
Request gets sent to the ajax view due to the client-side call making a
request to the url mapped to this view.

\item {} 
The form type is detected in the json data sent along with the call.

\item {} 
This string is then pulled out of the dict, executed and it's response
sent back to the browser.

\end{enumerate}
\begin{quote}\begin{description}
\item[{Parameters}] \leavevmode
\textbf{request} -- Automatically passed contains a map of the httprequest

\item[{Returns}] \leavevmode
HttpResponse object back to the browser.

\end{description}\end{quote}

\end{fulllineitems}

\index{edit\_profile() (in module timetracker.views)}

\begin{fulllineitems}
\phantomsection\label{timetracker:timetracker.views.edit_profile}\pysiglinewithargsret{\code{timetracker.views.}\bfcode{edit\_profile}}{\emph{request}, \emph{*args}, \emph{**kwargs}}{}
View for sending the user to the edit profile page

This view is a simple set of fields which allow all kinds of users to edit
pieces of information about their profile, currently it allows uers to
edit their name and their password.
\begin{quote}\begin{description}
\item[{Parameters}] \leavevmode
\textbf{request} -- Automatically passed contains a map of the httprequest

\item[{Returns}] \leavevmode
HttpResponse object back to the browser.

\end{description}\end{quote}

\end{fulllineitems}

\index{explain() (in module timetracker.views)}

\begin{fulllineitems}
\phantomsection\label{timetracker:timetracker.views.explain}\pysiglinewithargsret{\code{timetracker.views.}\bfcode{explain}}{\emph{request}, \emph{*args}, \emph{**kwargs}}{}
Renders the Balance explanation page

This page renders a simple template to show the users how their balance is
calculated. This view takes the user object, retrieves a couple of fields,
which are user.shiftlength and the associated values with that datetime
objects, constructs a string with them and passes it to the template as
the users `shiftlength' attribute. It then takes the count of working
days in the database so that the user has an idea of how many days they
have tracked altogether. Then it calculates their total balance and pushes
all these strings into the template.
\begin{quote}\begin{description}
\item[{Parameters}] \leavevmode
\textbf{request} -- Automatically passed contains a map of the httprequest

\item[{Returns}] \leavevmode
HttpResponse object back to the browser.

\end{description}\end{quote}

\end{fulllineitems}

\index{forgot\_pass() (in module timetracker.views)}

\begin{fulllineitems}
\phantomsection\label{timetracker:timetracker.views.forgot_pass}\pysiglinewithargsret{\code{timetracker.views.}\bfcode{forgot\_pass}}{\emph{request}}{}
Simple view for resetting a user's password

This view has a dual function. The first function is to simply render the
initial page which has a field and the themed markup. On this page a user
can enter their e-mail address and then click submit to have their
password sent to them.

The second function of this page is to respond to the change password
request. In the html markup of the `forgotpass.html' page you will see
that the intention is to have the page post to the same URL which this
page was rendered from. If the request contains POST information then we
retrieve that user from the database, construct an e-mail based on that
and send their password to them. Finally, we redirect to the login page.
\begin{quote}\begin{description}
\item[{Parameters}] \leavevmode
\textbf{request} -- Automatically passed contains a map of the httprequest

\item[{Returns}] \leavevmode
HttpResponse object back to the browser.

\end{description}\end{quote}

\end{fulllineitems}

\index{holiday\_planning() (in module timetracker.views)}

\begin{fulllineitems}
\phantomsection\label{timetracker:timetracker.views.holiday_planning}\pysiglinewithargsret{\code{timetracker.views.}\bfcode{holiday\_planning}}{\emph{request}, \emph{*args}, \emph{**kwargs}}{}
Generates the full holiday table for all employees under a manager

First we find the user object and find whether or not that user is a team
leader or not. If they are a team leader, which set a boolean flag to show
the template what kind of user is logged in. This is so that the team
leaders are not able to view certain things (e.g. Job Codes).

If the admin/tl tries to access the holiday page before any users have
been assigned to them, then we just throw them back to the main page. This
is doubly ensuring that they can't access what would otherwise be a
completely borked page.
\begin{quote}\begin{description}
\item[{Parameters}] \leavevmode
\textbf{request} -- Automatically passed contains a map of the httprequest

\item[{Returns}] \leavevmode
HttpResponse object back to the browser.

\end{description}\end{quote}

\end{fulllineitems}

\index{index() (in module timetracker.views)}

\begin{fulllineitems}
\phantomsection\label{timetracker:timetracker.views.index}\pysiglinewithargsret{\code{timetracker.views.}\bfcode{index}}{\emph{request}}{}
This function serves the base login page. TODO: Make this view check
to see if the user is already logged in and if so, redirect.

This function shouldn't be directly called, it's invocation is automatic
\begin{quote}\begin{description}
\item[{Parameters}] \leavevmode
\textbf{request} -- Automatically passed. Contains a map of the httprequest

\item[{Returns}] \leavevmode
A HttpResponse object which is then passed to the browser

\end{description}\end{quote}

\end{fulllineitems}

\index{login() (in module timetracker.views)}

\begin{fulllineitems}
\phantomsection\label{timetracker:timetracker.views.login}\pysiglinewithargsret{\code{timetracker.views.}\bfcode{login}}{\emph{request}}{}
This function logs the user in, directly adding the session id to
a database entry. This function is invoked from the url mapped in urls.py.
The url is POSTed to, and should contain two fields, the use\_name and the
pass word field. This is then pulled from the database and matched
against, what the user supplied. If they match, the user is then checked
to see what \emph{kind} of user their are, if they are ADMIN or TEAML they will
be sent to the administrator view. Else they will be sent to the user
page.

This function shouldn't be directly called, it's invocation is automatic
from the url mappings.
\begin{quote}\begin{description}
\item[{Parameters}] \leavevmode
\textbf{request} -- Automatically passed. Contains a map of the httprequest

\item[{Returns}] \leavevmode
A HttpResponse object which is then passed to the browser

\end{description}\end{quote}

\end{fulllineitems}

\index{logout() (in module timetracker.views)}

\begin{fulllineitems}
\phantomsection\label{timetracker:timetracker.views.logout}\pysiglinewithargsret{\code{timetracker.views.}\bfcode{logout}}{\emph{request}}{}
Simple logout function

This function will delete a session id from the session dictionary so that
the user will need to log back in order to access the same pages.
\begin{quote}\begin{description}
\item[{Parameters}] \leavevmode
\textbf{request} -- Automatically passed contains a map of the httprequest

\item[{Returns}] \leavevmode
A HttpResponse object which is passed to the browser.

\end{description}\end{quote}

\end{fulllineitems}

\index{user\_view() (in module timetracker.views)}

\begin{fulllineitems}
\phantomsection\label{timetracker:timetracker.views.user_view}\pysiglinewithargsret{\code{timetracker.views.}\bfcode{user\_view}}{\emph{request}, \emph{*args}, \emph{**kwargs}}{}
Generates a calendar based on the URL it receives.
For example: domain.com/calendar/\{year\}/\{month\}/\{day\},
also takes a day just in case you want to add a particular
view for a day, for example. Currently a day-level is not
in-use.
\begin{quote}\begin{description}
\item[{Note }] \leavevmode
The generated HTML should be pretty printed

\item[{Parameters}] \leavevmode\begin{itemize}
\item {} 
\textbf{request} -- Automatically passed contains a map of the httprequest

\item {} 
\textbf{year} -- The year that the view will be rendered with, default is
the current year.

\item {} 
\textbf{month} -- The month that the view will be rendered with, default is
the current month.

\item {} 
\textbf{day} -- The day that the view will be rendered with, default is
the current day

\end{itemize}

\item[{Returns}] \leavevmode
A HttpResponse object which is passed to the browser.

\end{description}\end{quote}

\end{fulllineitems}



\subsection{timetracker.utils.calendar\_utils}
\label{timetracker:timetracker-utils-calendar-utils}\label{timetracker:module-timetracker.utils.calendar_utils}\index{timetracker.utils.calendar\_utils (module)}
Module for collecting the utility functions dealing with mostly calendar
tasks, processing dates and creating time-based code.


\subsubsection{Module Functions}
\label{timetracker:module-functions}
\begin{tabulary}{\linewidth}{|L|L|}
\hline
\textbf{} & \textbf{}\\\hline

{\hyperref[timetracker:timetracker.utils.calendar_utils.get_request_data]{\code{get\_request\_data()}}}
 & 
{\hyperref[timetracker:timetracker.utils.calendar_utils.calendar_wrapper]{\code{calendar\_wrapper()}}}
\\\hline

{\hyperref[timetracker:timetracker.utils.calendar_utils.validate_time]{\code{validate\_time()}}}
 & 
{\hyperref[timetracker:timetracker.utils.calendar_utils.gen_holiday_list]{\code{gen\_holiday\_list()}}}
\\\hline

{\hyperref[timetracker:timetracker.utils.calendar_utils.parse_time]{\code{parse\_time()}}}
 & 
{\hyperref[timetracker:timetracker.utils.calendar_utils.ajax_add_entry]{\code{ajax\_add\_entry()}}}
\\\hline

{\hyperref[timetracker:timetracker.utils.calendar_utils.ajax_delete_entry]{\code{ajax\_delete\_entry()}}}
 & 
{\hyperref[timetracker:timetracker.utils.calendar_utils.ajax_error]{\code{ajax\_error()}}}
\\\hline

{\hyperref[timetracker:timetracker.utils.calendar_utils.ajax_change_entry]{\code{ajax\_change\_entry()}}}
 & 
{\hyperref[timetracker:timetracker.utils.calendar_utils.get_user_data]{\code{get\_user\_data()}}}
\\\hline

{\hyperref[timetracker:timetracker.utils.calendar_utils.delete_user]{\code{delete\_user()}}}
 & 
{\hyperref[timetracker:timetracker.utils.calendar_utils.useredit]{\code{useredit()}}}
\\\hline

{\hyperref[timetracker:timetracker.utils.calendar_utils.mass_holidays]{\code{mass\_holidays()}}}
 & 
{\hyperref[timetracker:timetracker.utils.calendar_utils.profile_edit]{\code{profile\_edit()}}}
\\\hline

{\hyperref[timetracker:timetracker.utils.calendar_utils.gen_datetime_cal]{\code{gen\_datetime\_cal()}}}
 & \\\hline
\end{tabulary}

\index{ajax\_add\_entry() (in module timetracker.utils.calendar\_utils)}

\begin{fulllineitems}
\phantomsection\label{timetracker:timetracker.utils.calendar_utils.ajax_add_entry}\pysiglinewithargsret{\code{timetracker.utils.calendar\_utils.}\bfcode{ajax\_add\_entry}}{\emph{request}}{}
Adds a calendar entry asynchronously.

This method is for RUSERs who wish to add a single entry to their
TrackingEntries. This method is only available via ajax and obviously
requires that users be logged in.

The client-side code which POSTs to this view should contain a json map
of, for example:

\begin{Verbatim}[commandchars=\\\{\}]
\PYG{n+nx}{json\PYGZus{}map} \PYG{o}{=} \PYG{p}{\PYGZob{}}
    \PYG{l+s+s1}{'entry\PYGZus{}date'}\PYG{o}{:} \PYG{l+s+s2}{"2012-01-01"}\PYG{p}{,}
    \PYG{l+s+s1}{'start\PYGZus{}time'}\PYG{o}{:} \PYG{l+s+s2}{"09:00"}\PYG{p}{,}
    \PYG{l+s+s1}{'end\PYGZus{}time'}\PYG{o}{:} \PYG{l+s+s2}{"17:00"}\PYG{p}{,}
    \PYG{l+s+s1}{'daytype'}\PYG{o}{:} \PYG{l+s+s2}{"WRKDY"}\PYG{p}{,}
    \PYG{l+s+s1}{'breaks'}\PYG{o}{:} \PYG{l+s+s2}{"00:15:00"}\PYG{p}{,}
\PYG{p}{\PYGZcb{}}
\end{Verbatim}

Consider that the UserID will be in the session database, then we simply
run some server-side validations and then enter the entry into the db,
there are also some client-side validation, which is essentially the same
as here. The redundancy for validation is just \emph{good practice} because of
the various malicious ways it is possible to subvert client-side
javascript or turn it off completely. Therefore, redundancy.

When this view is launched, we create a server-side counterpart of the
json which is in request object. We then fill it, passing None if there
are any items missing.

We then create a json\_data dict to store the json success/error codes in
to pass back to the User and inform them of the status of the ajax
request.

We then validate the data. Which involves only time validation.

The creation of the entry goes like this:
The form object holds purely the data that the TrackingEntry needs to
hold, it's also already validated, so, as insecure it looks, it's actually
perfectly fine as there has been client-side side validation and
server-side validation. There will also be validation on the database
level. So we can use **kwargs to instantiate the TrackingEntry and
.save() it without much worry for saving some erroneous and/or harmful
data.

If all goes well with saving the TrackingEntry, i.e. the entry isn't a
duplicate, or the database validation doesn't fail. We then generate the
calendar again using the entry\_date in the form. We use this date because
it's logical to assume that if the user enters a TrackingEntry using this
date, then their calendar will be showing this month.

We create the calendar and push it all back to the client. The client-side
code then updates the calendar display with the new data.
\begin{quote}\begin{description}
\item[{Parameters}] \leavevmode
\textbf{request} -- HttpRequest object.

\item[{Returns}] \leavevmode
\code{HttpResponse} object with the mime/application type as
json.

\item[{Return type}] \leavevmode
\code{HttpResponse}

\end{description}\end{quote}

\end{fulllineitems}

\index{ajax\_change\_entry() (in module timetracker.utils.calendar\_utils)}

\begin{fulllineitems}
\phantomsection\label{timetracker:timetracker.utils.calendar_utils.ajax_change_entry}\pysiglinewithargsret{\code{timetracker.utils.calendar\_utils.}\bfcode{ajax\_change\_entry}}{\emph{request}}{}
Changes a calendar entry asynchronously

This method works in an extremely similar fashion to {\hyperref[timetracker:timetracker.utils.calendar_utils.ajax_add_entry]{\code{ajax\_add\_entry()}}},
with modicum of difference. The main difference is that in the add\_entry
method, we are simply looking for the hidden-id and deleting it from the
table. In this method we are \emph{creating} an entry from the form object
and saving it into the table.
\begin{quote}\begin{description}
\item[{Parameters}] \leavevmode
\textbf{request} -- \code{HttpRequest}

\item[{Returns}] \leavevmode
\code{HttpResponse} with mime/application of JSON

\item[{Return type}] \leavevmode
\code{HttpResponse}

\end{description}\end{quote}

\end{fulllineitems}

\index{ajax\_delete\_entry() (in module timetracker.utils.calendar\_utils)}

\begin{fulllineitems}
\phantomsection\label{timetracker:timetracker.utils.calendar_utils.ajax_delete_entry}\pysiglinewithargsret{\code{timetracker.utils.calendar\_utils.}\bfcode{ajax\_delete\_entry}}{\emph{request}}{}
Asynchronously deletes an entry

This method is for RUSERs who wish to delete a single entry from
their TrackingEntries. This method is only available via ajax
and obviously requires that users be logged in.

We then create our json\_data map to hold our success status and any
error codes we may generate so that we may inform the user of the
status of the request once we complete.

This part of the code will catch all errors because, well, this is
production code and there's no chance I'll be letting server 500
errors bubble to the client without catching and making them
sound pretty and plausable. Therefore we catch all errors.

We then take the entry date, and generate the calendar for that year/
month.
\begin{quote}\begin{description}
\item[{Parameters}] \leavevmode
\textbf{request} -- \code{HttpRequest}

\item[{Returns}] \leavevmode
\code{HttpResponse} object with mime/application of json

\item[{Return type}] \leavevmode
\code{HttpResponse}

\end{description}\end{quote}

\end{fulllineitems}

\index{ajax\_error() (in module timetracker.utils.calendar\_utils)}

\begin{fulllineitems}
\phantomsection\label{timetracker:timetracker.utils.calendar_utils.ajax_error}\pysiglinewithargsret{\code{timetracker.utils.calendar\_utils.}\bfcode{ajax\_error}}{\emph{request}}{}
Returns a HttpResponse with JSON as a payload

This function is a simple way of instantiating an error when using
json\_functions. It is decorated with the json\_response decorator so that
the dict that we return is dumped into a json object.
\begin{quote}\begin{description}
\item[{Parameters}] \leavevmode
\textbf{error} -- \code{str} which contains the pretty
error, this will be seen by the user so
make sure it's understandable.

\item[{Returns}] \leavevmode
\code{HttpResponse} with mime/application of json.

\item[{Return type}] \leavevmode
\code{HttpResponse}

\end{description}\end{quote}

\end{fulllineitems}

\index{calendar\_wrapper() (in module timetracker.utils.calendar\_utils)}

\begin{fulllineitems}
\phantomsection\label{timetracker:timetracker.utils.calendar_utils.calendar_wrapper}\pysiglinewithargsret{\code{timetracker.utils.calendar\_utils.}\bfcode{calendar\_wrapper}}{\emph{function}}{}
Decorator which checks if the calendar function was
called as an ajax request or not, if so, then the
the wrapper constructs the arguments for the call
from the POST items
\begin{quote}\begin{description}
\item[{Parameters}] \leavevmode
\textbf{function} -- Literally just gen\_calendar.

\item[{Return type}] \leavevmode
Nothing directly because it returns gen\_calendar's

\end{description}\end{quote}

\end{fulllineitems}

\index{delete\_user() (in module timetracker.utils.calendar\_utils)}

\begin{fulllineitems}
\phantomsection\label{timetracker:timetracker.utils.calendar_utils.delete_user}\pysiglinewithargsret{\code{timetracker.utils.calendar\_utils.}\bfcode{delete\_user}}{\emph{request}}{}
Asynchronously deletes a user.

This function simply deletes a user. We asynchronously delete the user
because it provides a better user-experience for the people doing data
entry on the form. It also allows the page to not have to deal with a
jerky nor have to create annoying `loading' bars/spinners.
\begin{quote}\begin{description}
\item[{Note }] \leavevmode
This function should not be called directly.

\end{description}\end{quote}

This function should be POSTed to via an Ajax call. Like so:

\begin{Verbatim}[commandchars=\\\{\}]
\PYG{n+nx}{\PYGZdl{}}\PYG{p}{.}\PYG{n+nx}{ajaxSetup}\PYG{p}{(}\PYG{p}{\PYGZob{}}
    \PYG{n+nx}{type}\PYG{o}{:} \PYG{l+s+s2}{"POST"}\PYG{p}{,}
    \PYG{n+nx}{url}\PYG{o}{:} \PYG{l+s+s2}{"/ajax/"}\PYG{p}{,}       \PYG{c+c1}{// "ajax" is the url we created in urls.py}
    \PYG{n+nx}{dataType}\PYG{o}{:} \PYG{l+s+s2}{"json"}
\PYG{p}{\PYGZcb{}}\PYG{p}{)}\PYG{p}{;}

\PYG{n+nx}{\PYGZdl{}}\PYG{p}{.}\PYG{n+nx}{ajax}\PYG{p}{(}\PYG{p}{\PYGZob{}}
    \PYG{n+nx}{data}\PYG{o}{:} \PYG{p}{\PYGZob{}}
        \PYG{n+nx}{user\PYGZus{}id}\PYG{o}{:} \PYG{l+m+mi}{1}
    \PYG{p}{\PYGZcb{}}
\PYG{p}{\PYGZcb{}}\PYG{p}{)}\PYG{p}{;}
\end{Verbatim}

Once this is received, we check that the user POSTing this data is an
administrator, or at least a team leader and we go ahead and delete the
user from the table.
\begin{quote}\begin{description}
\item[{Parameters}] \leavevmode
\textbf{request} -- \code{HttpRequest}

\item[{Returns}] \leavevmode
\code{HttpResponse} mime/application JSON

\item[{Return type}] \leavevmode
\code{HttpResponse}

\end{description}\end{quote}

\end{fulllineitems}

\index{gen\_calendar() (in module timetracker.utils.calendar\_utils)}

\begin{fulllineitems}
\phantomsection\label{timetracker:timetracker.utils.calendar_utils.gen_calendar}\pysiglinewithargsret{\code{timetracker.utils.calendar\_utils.}\bfcode{gen\_calendar}}{\emph{*args}, \emph{**kwargs}}{}
Returns a HTML calendar, calling a database user to get their day-by-day
entries and gives each day a special CSS class so that days can be styled
individually.

How this works is that, we iterate through each of the entries found in the
TrackingEntry QuerySet for \{year\}/\{month\}. Create the table\textgreater{}td for that entry
then attach the CSS class to that td. This means that each different type of
day can be individually styled per the front-end style that is required.
The choice to use a custom calendar table is precisely \emph{because of} this fact
the jQueryUI calendar doesn't support the individual styling of days, nor does
it support event handling with the level of detail which we require.

Each day td has one of two functions assigned to it depending on whether the
day was an `empty' day or a non-empty day. The two functions are called:
\begin{quote}

\begin{Verbatim}[commandchars=\\\{\}]
\PYG{k+kd}{function} \PYG{n+nx}{toggleChangeEntries}\PYG{p}{(}\PYG{n+nx}{st\PYGZus{}hour}\PYG{p}{,} \PYG{n+nx}{st\PYGZus{}min}\PYG{p}{,} \PYG{n+nx}{full\PYGZus{}st}\PYG{p}{,}
                             \PYG{n+nx}{fi\PYGZus{}hour}\PYG{p}{,} \PYG{n+nx}{fi\PYGZus{}min}\PYG{p}{,} \PYG{n+nx}{full\PYGZus{}fi}\PYG{p}{,}
                             \PYG{n+nx}{entry\PYGZus{}date}\PYG{p}{,} \PYG{n+nx}{daytype}\PYG{p}{,}
                             \PYG{n+nx}{change\PYGZus{}id}\PYG{p}{,} \PYG{n+nx}{breakLength}\PYG{p}{,}
                             \PYG{n+nx}{breakLength\PYGZus{}full}\PYG{p}{)}
 \PYG{c+c1}{// and}

 \PYG{k+kd}{function} \PYG{n+nx}{hideEntries}\PYG{p}{(}\PYG{n+nx}{date}\PYG{p}{)}
\end{Verbatim}
\end{quote}

These two functions could be slightly more generically named, as the calendar
markup is used in two different places, in the \{templates\}/calendar.html and
the \{templates\}/admin\_view.html therefore I will move to naming these based
on their event names, i.e. `calendarClickEventDay()' and
`calendarClickEventEmpty'.

The toggleChangeEntries() function takes 11 arguments, yes. 11. It's quite
a lot but it's all the relevant data associated with a tracking entry.
\begin{enumerate}
\item {} 
st\_hour is the start hour of the tracking entry, just the hour.

\item {} 
st\_min is the start minute of the tracking entry, just the minute.

\item {} 
full\_st is the full start time of the tracking entry.

\item {} 
fi\_hour is the end hour of the tracking entry, just the hour.

\item {} 
fi\_min is the end minute of the tracking entry, just the minute.

\item {} 
full\_fi is the full end time of the tracking entry.

\item {} 
entry\_date is the entry date of the tracking entry.

\item {} 
daytype is the daytype of the tracking entry.

\item {} 
change\_id this is the ID of the tracking entry.

\item {} 
breakLength this is the break length's minutes. Such as `15'.

\item {} 
This is the breaklength string such as ``00:15:00''

\end{enumerate}

The hideEntries function takes a single parameter, date which is the date of
the entry you want to fill in the Add Entry form.

The generated HTML should be `pretty printed' as well, so the output code
should be pretty readable.
\begin{quote}\begin{description}
\item[{Parameters}] \leavevmode\begin{itemize}
\item {} 
\textbf{year} -- Integer for the year required for output, defaults to the
current year.

\item {} 
\textbf{month} -- Integer for the month required for output, defaults to the
current month.

\item {} 
\textbf{day} -- Integer for the day required for output, defaults to the
current day.

\item {} 
\textbf{user} -- Integer ID for the user in the database, this will automatically,
be passed to this function. However, if you need to use it in
another setting make sure this is passed.

\end{itemize}

\item[{Returns}] \leavevmode
HTML String

\end{description}\end{quote}

\end{fulllineitems}

\index{gen\_datetime\_cal() (in module timetracker.utils.calendar\_utils)}

\begin{fulllineitems}
\phantomsection\label{timetracker:timetracker.utils.calendar_utils.gen_datetime_cal}\pysiglinewithargsret{\code{timetracker.utils.calendar\_utils.}\bfcode{gen\_datetime\_cal}}{\emph{year}, \emph{month}}{}
Generates a datetime list of all days in a month
\begin{quote}\begin{description}
\item[{Parameters}] \leavevmode\begin{itemize}
\item {} 
\textbf{year} -- \code{int}

\item {} 
\textbf{month} -- \code{int}

\end{itemize}

\item[{Returns}] \leavevmode
A flat list of datetime objects for the given month

\item[{Return type}] \leavevmode
\code{List} containing \code{datetime.datetime} objects.

\end{description}\end{quote}

\end{fulllineitems}

\index{gen\_holiday\_list() (in module timetracker.utils.calendar\_utils)}

\begin{fulllineitems}
\phantomsection\label{timetracker:timetracker.utils.calendar_utils.gen_holiday_list}\pysiglinewithargsret{\code{timetracker.utils.calendar\_utils.}\bfcode{gen\_holiday\_list}}{\emph{admin\_user}, \emph{year=2012}, \emph{month=5}}{}
Outputs a holiday calendar for that month.

For each user we get their tracking entries, then iterate over each of
their entries checking if it is a holiday or not, if it is then we change
the class entry for that number in the day class' dict. Adds a submit
button along with passing the user\_id to it.
\begin{quote}\begin{description}
\item[{Parameters}] \leavevmode\begin{itemize}
\item {} 
\textbf{admin\_user} -- {\hyperref[timetracker:timetracker.tracker.models.Tbluser]{\code{timetracker.tracker.models.Tbluser}}} instance.

\item {} 
\textbf{year} -- \code{int} of the year required to be output, defaults to
the current year.

\item {} 
\textbf{month} -- \code{int} of the month required to be output, defaults to
the current month.

\end{itemize}

\item[{Returns}] \leavevmode
A partially pretty printed html string.

\item[{Return type}] \leavevmode
\code{str}

\end{description}\end{quote}

\end{fulllineitems}

\index{get\_request\_data() (in module timetracker.utils.calendar\_utils)}

\begin{fulllineitems}
\phantomsection\label{timetracker:timetracker.utils.calendar_utils.get_request_data}\pysiglinewithargsret{\code{timetracker.utils.calendar\_utils.}\bfcode{get\_request\_data}}{\emph{form}, \emph{request}}{}
Given a form and a request object we pull out
from the request what the form defines.

i.e.:

\begin{Verbatim}[commandchars=\\\{\}]
\PYG{n}{form} \PYG{o}{=} \PYG{p}{\PYGZob{}}
     \PYG{l+s}{'}\PYG{l+s}{data1}\PYG{l+s}{'}\PYG{p}{:} \PYG{n+nb+bp}{None}
\PYG{p}{\PYGZcb{}}
\end{Verbatim}

get\_request\_data(form, request) will then fill that data with what's in
the request object.
\begin{quote}\begin{description}
\item[{Parameters}] \leavevmode\begin{itemize}
\item {} 
\textbf{form} -- A dictionary of items which should be filled from

\item {} 
\textbf{request} -- The request object where the data should be taken from.

\end{itemize}

\item[{Returns}] \leavevmode
A dictionary which contains the actual data from the request.

\item[{Return type}] \leavevmode
\code{dict}

\item[{Raises }] \leavevmode
KeyError

\end{description}\end{quote}

\end{fulllineitems}

\index{get\_user\_data() (in module timetracker.utils.calendar\_utils)}

\begin{fulllineitems}
\phantomsection\label{timetracker:timetracker.utils.calendar_utils.get_user_data}\pysiglinewithargsret{\code{timetracker.utils.calendar\_utils.}\bfcode{get\_user\_data}}{\emph{request}}{}
Returns a user as a json object.

This is a very simple method. First, the \code{HttpRequest} POST is
checked to see if it contains a user\_id. If so, we grab that user from
the database and take all their relevant information and encode it into
JSON then send it back to the browser.
\begin{quote}\begin{description}
\item[{Parameters}] \leavevmode
\textbf{request} -- \code{HttpRequest} object

\item[{Returns}] \leavevmode
\code{HttpRequest} with mime/application of JSON

\item[{Return type}] \leavevmode
\code{HttpResponse}

\end{description}\end{quote}

\end{fulllineitems}

\index{mass\_holidays() (in module timetracker.utils.calendar\_utils)}

\begin{fulllineitems}
\phantomsection\label{timetracker:timetracker.utils.calendar_utils.mass_holidays}\pysiglinewithargsret{\code{timetracker.utils.calendar\_utils.}\bfcode{mass\_holidays}}{\emph{request}}{}
Adds a holidays for a specific user en masse

This function takes a large amount of holidays as json input, iterates
over them, adding or deleting each one from the database.

The json data looks as such:

\begin{Verbatim}[commandchars=\\\{\}]
\PYG{n+nx}{holidays} \PYG{o}{=} \PYG{p}{\PYGZob{}}
    \PYG{l+m+mi}{1}\PYG{o}{:} \PYG{n+nx}{daytype}\PYG{p}{,}
    \PYG{l+m+mi}{2}\PYG{o}{:} \PYG{n+nx}{daytype}\PYG{p}{,}
    \PYG{l+m+mi}{3}\PYG{o}{:} \PYG{n+nx}{daytype}
    \PYG{p}{.}\PYG{p}{.}\PYG{p}{.}
 \PYG{p}{\PYGZcb{}}
\end{Verbatim}

And so on, for the entire month. In the request object we also have the
month and the year. We use this to create a date to filter the month by,
this is so that we're not deleting/changing the wrong month. The
year/month are taken from the current table headings on the client. We
then check what kind of day it is.

If the daytype is `empty' then we attempt to retrieve the day mapped to
that date, if there's an entry, we delete it. This is because when the
holiday page is rendered it shows whether or not that day is assigned. If
it was assigned and now it's empty, it means the user has marked it as
empty.

If the daytype is \emph{not} empty, then we create a new TrackingEntry instance
using the data that was the current step of iteration through the
holiday\_data dict. This will be a number and a daytype. We have the user
we're uploading this for and the year/month from the request object. We
also choose sensible defaults for what we're not supplied with, i.e. we're
not supplied with start/end times, nor a break time. This is because the
holiday page only deals with \emph{non-working-days} therefore we can track
these days with zeroed times.

If at this point an IntegrityError is raised, it means one of two things:
we can either have a duplicate entry, in which case we retrieve that entry
and change it's daytype, or we can have a different error, in which case
we wrap up working with this set of data and return an error to the
browser.

If all goes well, we mark the return object's success attribute with True
and return.
\begin{quote}\begin{description}
\item[{Parameters}] \leavevmode
\textbf{request} -- \code{HttpRequest}

\item[{Returns}] \leavevmode
\code{HttpResponse} with mime/application as JSON

\item[{Note }] \leavevmode
All exceptions are caught, however here is a list:

\item[{Raises }] \leavevmode
\code{IntegrityError} \code{DoesNotExist}
\code{ValidationError} \code{Exception}

\end{description}\end{quote}

\end{fulllineitems}

\index{parse\_time() (in module timetracker.utils.calendar\_utils)}

\begin{fulllineitems}
\phantomsection\label{timetracker:timetracker.utils.calendar_utils.parse_time}\pysiglinewithargsret{\code{timetracker.utils.calendar\_utils.}\bfcode{parse\_time}}{\emph{timestring}, \emph{type\_of=\textless{}type `int'\textgreater{}}}{}
Given a time string will return a tuple of ints,
i.e. ``09:44'' returns {[}9, 44{]} with the default args,
you can pass any function to the type argument.
\begin{quote}\begin{description}
\item[{Parameters}] \leavevmode\begin{itemize}
\item {} 
\textbf{timestring} -- String such as `09:44'

\item {} 
\textbf{type\_of} -- A type which the split string should be converted to,
suitable types are: \code{int}, \code{str} and
\code{float}.

\end{itemize}

\end{description}\end{quote}

\end{fulllineitems}

\index{profile\_edit() (in module timetracker.utils.calendar\_utils)}

\begin{fulllineitems}
\phantomsection\label{timetracker:timetracker.utils.calendar_utils.profile_edit}\pysiglinewithargsret{\code{timetracker.utils.calendar\_utils.}\bfcode{profile\_edit}}{\emph{request}}{}
Asynchronously edits a user's profile.

Access Level: All

First we pull out the user instance that is currently logged in. Then as
with most ajax functions, we construct a map to receive what should be in
the in the POST object. This view specifically deals with changing a Name,
Surname and Password. Any other data is not required to be changed.

Once this data has been populated from the POST object we then retrieve
the string names for the attributes and use setattr to change them to what
we've been supplied here.
\begin{quote}\begin{description}
\item[{Parameters}] \leavevmode
\textbf{request} -- \code{HttpRequest}

\item[{Returns}] \leavevmode
\code{HttpResponse} with mime/application as JSON

\end{description}\end{quote}

\end{fulllineitems}

\index{useredit() (in module timetracker.utils.calendar\_utils)}

\begin{fulllineitems}
\phantomsection\label{timetracker:timetracker.utils.calendar_utils.useredit}\pysiglinewithargsret{\code{timetracker.utils.calendar\_utils.}\bfcode{useredit}}{\emph{request}}{}
This function both adds and edits a user
\begin{itemize}
\item {} 
Adding a user

\end{itemize}

Adding a user via ajax. This function cannot be used outside of an ajax
request. This is simply because there's no need. If there ever is a need
to synchronously add users then I will remove the @request\_check from the
function.

The function shouldn't be called directly, instead, you should POST to the
ajax view which points to this via \code{timetracker.urls} you also
need to include in the POST data. Here is an example call using jQuery:

\begin{Verbatim}[commandchars=\\\{\}]
\PYG{n+nx}{\PYGZdl{}}\PYG{p}{.}\PYG{n+nx}{ajaxSetup}\PYG{p}{(}\PYG{p}{\PYGZob{}}
    \PYG{n+nx}{type}\PYG{o}{:} \PYG{l+s+s2}{"POST"}\PYG{p}{,}
    \PYG{n+nx}{dataType}\PYG{o}{:} \PYG{l+s+s2}{"json"}
\PYG{p}{\PYGZcb{}}\PYG{p}{)}\PYG{p}{;}

\PYG{n+nx}{\PYGZdl{}}\PYG{p}{.}\PYG{n+nx}{ajax}\PYG{p}{(}\PYG{p}{\PYGZob{}}
    \PYG{n+nx}{url}\PYG{o}{:} \PYG{l+s+s2}{"/ajax/"}\PYG{p}{,}
    \PYG{n+nx}{data}\PYG{o}{:} \PYG{p}{\PYGZob{}}
        \PYG{l+s+s1}{'user\PYGZus{}id'}\PYG{o}{:} \PYG{l+s+s2}{"aaron.france@hp.com"}\PYG{p}{,}
        \PYG{l+s+s1}{'firstname'}\PYG{o}{:} \PYG{l+s+s2}{"Aaron"}\PYG{p}{,}
        \PYG{l+s+s1}{'lastname'}\PYG{o}{:} \PYG{l+s+s2}{"France"}\PYG{p}{,}
        \PYG{l+s+s1}{'user\PYGZus{}type'}\PYG{o}{:} \PYG{l+s+s2}{"RUSER"}\PYG{p}{,}
        \PYG{l+s+s1}{'market'}\PYG{o}{:} \PYG{l+s+s2}{"BK"}\PYG{p}{,}
        \PYG{l+s+s1}{'process'}\PYG{o}{:} \PYG{l+s+s2}{"AR"}\PYG{p}{,}
        \PYG{l+s+s1}{'start\PYGZus{}date'}\PYG{o}{:} \PYG{l+s+s2}{"2012-01-01"}
        \PYG{l+s+s1}{'breaklength'}\PYG{o}{:} \PYG{l+s+s2}{"00:15:00"}
        \PYG{l+s+s1}{'shiftlength'}\PYG{o}{:} \PYG{l+s+s2}{"00:07:45"}
        \PYG{l+s+s1}{'job\PYGZus{}code'}\PYG{o}{:} \PYG{l+s+s2}{"ABC123"}
        \PYG{l+s+s1}{'holiday\PYGZus{}balance'}\PYG{o}{:} \PYG{l+m+mi}{20}\PYG{p}{,}
        \PYG{l+s+s1}{'mode'}\PYG{o}{:} \PYG{l+s+s2}{"false"}
    \PYG{p}{\PYGZcb{}}
\PYG{p}{\PYGZcb{}}\PYG{p}{)}\PYG{p}{;}
\end{Verbatim}

You would also create success and error handlers but for the sake of
documentation lets assume you know what you're doing with javascript. When
the function receives this data, it first checks the `mode' attribute of
the json data. If it contains `false' then we are looking at an `add\_user'
kind of request. Because of this, and the client-side validation that is
done. We simply use some **kwargs magic on the
{\hyperref[timetracker:timetracker.tracker.models.Tbluser]{\code{timetracker.tracker.models.Tbluser}}} constructor and save our
Tbluser object.

Providing that this didn't throw an error and it may, the next step is to
create a Tblauthorization link to make sure that the user that created
this user instance has the newly created user assigned to their team (or
to their manager's team in the case of team leaders). We make the team
leader check, if it's a team leader we call get\_administrator() on the
authorized user and then save the newly created user into the
Tblauthorization instance found. Once this has happened we send the user
an e-mail informing them of their account details and the password that we
generated for them.
\begin{itemize}
\item {} 
Editing a user

\end{itemize}

This function also deals with the \emph{editing} of a user instance, it's
possible that this functionality will be refactored into it's own function
but for now, we have both in here.

Editing a user happens much the same as adding a user save for some very
minor differences:
\begin{quote}

\begin{Verbatim}[commandchars=\\\{\}]
\PYG{n+nx}{\PYGZdl{}}\PYG{p}{.}\PYG{n+nx}{ajaxSetup}\PYG{p}{(}\PYG{p}{\PYGZob{}}
   \PYG{n+nx}{type}\PYG{o}{:} \PYG{l+s+s2}{"POST"}\PYG{p}{,}
   \PYG{n+nx}{dataType}\PYG{o}{:} \PYG{l+s+s2}{"json"}
\PYG{p}{\PYGZcb{}}\PYG{p}{)}\PYG{p}{;}

\PYG{n+nx}{\PYGZdl{}}\PYG{p}{.}\PYG{n+nx}{ajax}\PYG{p}{(}\PYG{p}{\PYGZob{}}
    \PYG{n+nx}{url}\PYG{o}{:} \PYG{l+s+s2}{"/ajax/"}\PYG{p}{,}
    \PYG{n+nx}{data}\PYG{o}{:} \PYG{p}{\PYGZob{}}
        \PYG{l+s+s1}{'user\PYGZus{}id'}\PYG{o}{:} \PYG{l+s+s2}{"aaron.france@hp.com"}\PYG{p}{,}
        \PYG{l+s+s1}{'firstname'}\PYG{o}{:} \PYG{l+s+s2}{"Aaron"}\PYG{p}{,}
        \PYG{l+s+s1}{'lastname'}\PYG{o}{:} \PYG{l+s+s2}{"France"}\PYG{p}{,}
        \PYG{l+s+s1}{'user\PYGZus{}type'}\PYG{o}{:} \PYG{l+s+s2}{"RUSER"}\PYG{p}{,}
        \PYG{l+s+s1}{'job\PYGZus{}code'}\PYG{o}{:} \PYG{l+s+s2}{"ABC456"}
        \PYG{l+s+s1}{'holiday\PYGZus{}balance'}\PYG{o}{:} \PYG{l+m+mi}{50}\PYG{p}{,}
        \PYG{l+s+s1}{'mode'}\PYG{o}{:} \PYG{l+m+mi}{1}
    \PYG{p}{\PYGZcb{}}
\PYG{p}{\PYGZcb{}}\PYG{p}{)}\PYG{p}{;}
\end{Verbatim}
\end{quote}

You may notice that the amount of data isn't the same. When editing a user
it is not vital that all attributes of the user instance are changed
and/or sent to this view. This is because of the method used to assign
back to the user instance the changes of attributes (getattr/setattr).

The attribute which determines that the call is an edit call and not a add
user call is the mode, if the mode is not false and is a number.

When we first step into this function we look for the mode attribute of
the json data. If it's a number then we look up the user with that user\_id
we then step through each attribute on the request map and assign it to
the user object which we retrieved from the database.
\begin{quote}\begin{description}
\item[{Parameters}] \leavevmode
\textbf{request} -- \code{HttpRequest}

\item[{Returns}] \leavevmode
\code{HttpResponse} with mime/application of JSON

\item[{Raises }] \leavevmode
\code{Integrity} \code{Validation} and \code{Exception}

\item[{Note }] \leavevmode
Please remember that all exceptions are caught here and to make
sure that things are working be sure to read the response in the
browser to see if there are any errors.

\end{description}\end{quote}

\end{fulllineitems}

\index{validate\_time() (in module timetracker.utils.calendar\_utils)}

\begin{fulllineitems}
\phantomsection\label{timetracker:timetracker.utils.calendar_utils.validate_time}\pysiglinewithargsret{\code{timetracker.utils.calendar\_utils.}\bfcode{validate\_time}}{\emph{start}, \emph{end}}{}
Validates that the start time is before the end time
\begin{quote}\begin{description}
\item[{Parameters}] \leavevmode\begin{itemize}
\item {} 
\textbf{start} -- String time such as ``09:45''

\item {} 
\textbf{end} -- String time such as ``17:00''

\end{itemize}

\item[{Return type}] \leavevmode
\code{boolean}

\end{description}\end{quote}

\end{fulllineitems}



\subsection{timetracker.utils.datemaps}
\label{timetracker:timetracker-utils-datemaps}\label{timetracker:module-timetracker.utils.datemaps}\index{timetracker.utils.datemaps (module)}
Maps of useful data

Django has several of these built-in but they are annoying to use.

\code{WEEK\_MAP\_MID}: This is a map of the days of the week along with the
mid-length string for that value. For example:

\begin{Verbatim}[commandchars=\\\{\}]
\PYG{n}{WEEK\PYGZus{}MAP\PYGZus{}MID} \PYG{o}{=} \PYG{p}{\PYGZob{}}
    \PYG{l+m+mi}{0}\PYG{p}{:} \PYG{l+s}{'}\PYG{l+s}{Mon}\PYG{l+s}{'}\PYG{p}{,}
    \PYG{l+m+mi}{1}\PYG{p}{:} \PYG{l+s}{'}\PYG{l+s}{Tue}\PYG{l+s}{'}\PYG{p}{,}
    \PYG{o}{.}\PYG{o}{.}\PYG{o}{.}
\PYG{p}{\PYGZcb{}}
\end{Verbatim}

\code{WEEK\_MAP\_SHORT}: This is similar except using a longer string.

\code{MONTH\_MAP}: This is a map of the months which refer to a two-element
tuple which has the short code for the month and the long string for that
month. I.e. `JAN' and `January'.

\code{WORKING\_CHOICES}: This is a tuple of two-element tuples which contain
the only possible working day possibilites.

\code{ABSENT\_CHOICES}: This is a tuple of two-element tuples which contain
the only possible absent day possibilities.

\code{DAYTYPE\_CHOICES}: This is both \code{WORKING\_CHOICES} and
\code{ABSENT\_CHOICES} joined together to give all the daytype possibilities.
\index{float\_to\_time() (in module timetracker.utils.datemaps)}

\begin{fulllineitems}
\phantomsection\label{timetracker:timetracker.utils.datemaps.float_to_time}\pysiglinewithargsret{\code{timetracker.utils.datemaps.}\bfcode{float\_to\_time}}{\emph{timefloat}}{}
Takes a float and returns the same representation of time.
\begin{quote}\begin{description}
\item[{Parameters}] \leavevmode
\textbf{timefloat} -- This is a \code{float} which needs to be represented
as a timestring.

\item[{Return type}] \leavevmode
\code{str} such as `00:12' or `09:15'

\end{description}\end{quote}

\end{fulllineitems}

\index{generate\_select() (in module timetracker.utils.datemaps)}

\begin{fulllineitems}
\phantomsection\label{timetracker:timetracker.utils.datemaps.generate_select}\pysiglinewithargsret{\code{timetracker.utils.datemaps.}\bfcode{generate\_select}}{\emph{data}, \emph{id='`}}{}
Generates a select box from a tuple of tuples

\begin{Verbatim}[commandchars=\\\{\}]
\PYG{n}{generate\PYGZus{}select}\PYG{p}{(}\PYG{p}{(}
    \PYG{p}{(}\PYG{l+s}{'}\PYG{l+s}{val1}\PYG{l+s}{'}\PYG{p}{,} \PYG{l+s}{'}\PYG{l+s}{Value One}\PYG{l+s}{'}\PYG{p}{)}\PYG{p}{,}
    \PYG{p}{(}\PYG{l+s}{'}\PYG{l+s}{val2}\PYG{l+s}{'}\PYG{p}{,} \PYG{l+s}{'}\PYG{l+s}{Value Two}\PYG{l+s}{'}\PYG{p}{)}\PYG{p}{,}
    \PYG{p}{(}\PYG{l+s}{'}\PYG{l+s}{val3}\PYG{l+s}{'}\PYG{p}{,} \PYG{l+s}{'}\PYG{l+s}{Value Three}\PYG{l+s}{'}\PYG{p}{)}
\PYG{p}{)}\PYG{p}{)}
\end{Verbatim}

will return:-

\begin{Verbatim}[commandchars=\\\{\}]
\PYG{n+nt}{\PYGZlt{}select} \PYG{n+na}{id=}\PYG{l+s}{''}\PYG{n+nt}{\PYGZgt{}}
   \PYG{n+nt}{\PYGZlt{}option} \PYG{n+na}{value=}\PYG{l+s}{"val1"}\PYG{n+nt}{\PYGZgt{}}Value One\PYG{n+nt}{\PYGZlt{}/option\PYGZgt{}}
   \PYG{n+nt}{\PYGZlt{}option} \PYG{n+na}{value=}\PYG{l+s}{"val2"}\PYG{n+nt}{\PYGZgt{}}Value Two\PYG{n+nt}{\PYGZlt{}/option\PYGZgt{}}
   \PYG{n+nt}{\PYGZlt{}option} \PYG{n+na}{value=}\PYG{l+s}{"val3"}\PYG{n+nt}{\PYGZgt{}}Value Three\PYG{n+nt}{\PYGZlt{}/option\PYGZgt{}}
\PYG{n+nt}{\PYGZlt{}/select\PYGZgt{}}
\end{Verbatim}
\begin{quote}\begin{description}
\item[{Parameters}] \leavevmode
\textbf{data} -- This is a tuple of tuples (can also be a list of lists. But
tuples will behave more efficiently than lists and who likes
mutation anyway?

\item[{Return type}] \leavevmode
\code{str}/HTML

\end{description}\end{quote}

\end{fulllineitems}

\index{pad() (in module timetracker.utils.datemaps)}

\begin{fulllineitems}
\phantomsection\label{timetracker:timetracker.utils.datemaps.pad}\pysiglinewithargsret{\code{timetracker.utils.datemaps.}\bfcode{pad}}{\emph{string}, \emph{padchr=`0'}, \emph{amount=2}}{}
Pads a string
\begin{quote}\begin{description}
\item[{Parameters}] \leavevmode\begin{itemize}
\item {} 
\textbf{string} -- This is the string you want to pad.

\item {} 
\textbf{padchr} -- This is the character you want to pad the string with.

\item {} 
\textbf{amount} -- This is the length of the string you want the input end up.

\end{itemize}

\item[{Return type}] \leavevmode
\code{str}

\end{description}\end{quote}

\end{fulllineitems}



\subsection{timetracker.utils.error\_codes}
\label{timetracker:timetracker-utils-error-codes}\label{timetracker:module-timetracker.utils.error_codes}\index{timetracker.utils.error\_codes (module)}
Database errors give out a tuple of information. The first of which is a
number, we can grab that and check what kind of error we're getting

Currently this is somewhat small as there aren't many errors that get thrown
consistantly to warrant putting their codes into a module.

\code{DUPLICATE\_ENTRY}: This is thrown when the database validation reports
that there is already an entry in the database with conflicting values for
whatever is set as a Unique and/or UniqueTogether.

\code{CONNECTION\_REFUSED}: This error is thrown when the SMTP server is down
and/or is not responding.


\subsection{timetracker.utils.decorators}
\label{timetracker:timetracker-utils-decorators}\label{timetracker:module-timetracker.utils.decorators}\index{timetracker.utils.decorators (module)}
Module to for sharing decorators between all modules
\index{admin\_check() (in module timetracker.utils.decorators)}

\begin{fulllineitems}
\phantomsection\label{timetracker:timetracker.utils.decorators.admin_check}\pysiglinewithargsret{\code{timetracker.utils.decorators.}\bfcode{admin\_check}}{\emph{func}}{}
Wrapper to see if the view is being called as an admin

This works by 1) Checking if there is a user\_id in the session
table. 2) If that user is a real user in the database and 3) if
that user's is\_admin() returns True.
\begin{quote}\begin{description}
\item[{Parameters}] \leavevmode
\textbf{func} -- A function with a request object as a parameter

\item[{Returns}] \leavevmode
Nothing directly, it returns the function it decorates.

\item[{Raises }] \leavevmode
\code{Http404} error

\end{description}\end{quote}

\end{fulllineitems}

\index{json\_response() (in module timetracker.utils.decorators)}

\begin{fulllineitems}
\phantomsection\label{timetracker:timetracker.utils.decorators.json_response}\pysiglinewithargsret{\code{timetracker.utils.decorators.}\bfcode{json\_response}}{\emph{func}}{}
Decorator function that when applied to a function which
returns some json data will be turned into a HttpResponse

This is useful because the call site can literally just
call the function as it is without needed to make a http-
response.
\begin{quote}\begin{description}
\item[{Parameters}] \leavevmode
\textbf{func} -- Function which returns a dictionary

\item[{Returns}] \leavevmode
\code{HttpResponse} with mime/application as JSON

\end{description}\end{quote}

\end{fulllineitems}

\index{loggedin() (in module timetracker.utils.decorators)}

\begin{fulllineitems}
\phantomsection\label{timetracker:timetracker.utils.decorators.loggedin}\pysiglinewithargsret{\code{timetracker.utils.decorators.}\bfcode{loggedin}}{\emph{func}}{}
Decorator to make sure that the view is being accessed by a
logged in user.

This works by simply checking that the user\_id in the session
table is 1) There and 2) A real user. If either of these aren't
satisfied we throw back a 404.

We also log this.
\begin{quote}\begin{description}
\item[{Parameters}] \leavevmode
\textbf{func} -- A function which has a request object as a parameter

\item[{Returns}] \leavevmode
Nothing directly, it returns the function it decorates

\item[{Raises }] \leavevmode
\code{Http404} error

\end{description}\end{quote}

\end{fulllineitems}

\index{request\_check() (in module timetracker.utils.decorators)}

\begin{fulllineitems}
\phantomsection\label{timetracker:timetracker.utils.decorators.request_check}\pysiglinewithargsret{\code{timetracker.utils.decorators.}\bfcode{request\_check}}{\emph{func}}{}
Decorator to check an incoming request against a few rules

This function should decorate any function which is supposed
to be accessed by and only by Ajax. If we see that the function
was accessed by any other means, we raise a \code{Http404}
and give up processing the page.

We also make a redundant check to see if the user is logged in.
\begin{quote}\begin{description}
\item[{Parameters}] \leavevmode
\textbf{func} -- Function which has a request object parameter.

\item[{Returns}] \leavevmode
The function which it decorates.

\item[{Raises }] \leavevmode
\code{Http404}

\end{description}\end{quote}

\end{fulllineitems}



\subsection{timetracker.tracker.models}
\label{timetracker:module-timetracker.tracker.models}\label{timetracker:timetracker-tracker-models}\index{timetracker.tracker.models (module)}
Definition of the models used in the timetracker app
\begin{quote}\begin{description}
\item[{platform}] \leavevmode
All

\item[{synopsis}] \leavevmode
Module which contains view functions that are mapped from urls

\end{description}\end{quote}
\index{Tblauthorization (class in timetracker.tracker.models)}

\begin{fulllineitems}
\phantomsection\label{timetracker:timetracker.tracker.models.Tblauthorization}\pysiglinewithargsret{\strong{class }\code{timetracker.tracker.models.}\bfcode{Tblauthorization}}{\emph{*args}, \emph{**kwargs}}{}
Links Administrators (managers) with their team.

This table is a many-to-many relationship between Administrators and any
other any user type in TEAML/RUSER.

This table is used to explicitly show which people are in an
Administrator's team. Usually in SQL-land, you would be able to
instanstiate multiple rows of many-to-many relationships, however, due to
the fact that working with these tables in an object orientated fashion is
far simpler adding multiple relationships to the same
{\hyperref[timetracker:timetracker.tracker.models.Tblauthorization]{\code{Tblauthorization}}} object, we re-use the relationship when
creating/adding additional {\hyperref[timetracker:timetracker.tracker.models.Tblauthorization]{\code{Tblauthorization}}} instances.

This means that, if you were to need to add a relationship between an
Administrator and a RUSER, then you would need to make sure that you
retrieve the {\hyperref[timetracker:timetracker.tracker.models.Tblauthorization]{\code{Tblauthorization}}} object \emph{before} and save the new
link using that instance. Failure to do this would mean that areas where
the .get() method is employed would start to throw
Tblauthorization.MultipleObjectsReturned.

In future, and time, I would like to make it so that the .save() method is
overloaded and then we can check if a {\hyperref[timetracker:timetracker.tracker.models.Tblauthorization]{\code{Tblauthorization}}} link
already exists and if so, save to that instead.
\index{display\_users() (timetracker.tracker.models.Tblauthorization method)}

\begin{fulllineitems}
\phantomsection\label{timetracker:timetracker.tracker.models.Tblauthorization.display_users}\pysiglinewithargsret{\bfcode{display\_users}}{}{}
Method which generates the HTML for the admin views

This method is depracated in favour of not actually using the admin
interface to interact with {\hyperref[timetracker:timetracker.tracker.models.Tblauthorization]{\code{Tblauthorization}}} instances too
much. That and, it's not unicode-safe.
\begin{quote}\begin{description}
\item[{Return type}] \leavevmode
\code{string}

\end{description}\end{quote}

\end{fulllineitems}

\index{manager\_view() (timetracker.tracker.models.Tblauthorization method)}

\begin{fulllineitems}
\phantomsection\label{timetracker:timetracker.tracker.models.Tblauthorization.manager_view}\pysiglinewithargsret{\bfcode{manager\_view}}{}{}
Method which negates needing to retrieve the users via the
Tblauthorization.objects.all() method which is, needless to say, a
mouthfull.
\begin{quote}\begin{description}
\item[{Return type}] \leavevmode
\code{QuerySet}

\end{description}\end{quote}

\end{fulllineitems}

\index{teamleader\_view() (timetracker.tracker.models.Tblauthorization method)}

\begin{fulllineitems}
\phantomsection\label{timetracker:timetracker.tracker.models.Tblauthorization.teamleader_view}\pysiglinewithargsret{\bfcode{teamleader\_view}}{}{}
Method which provides a shortcut to retrieving the set of users which
are available to the TeamLeader based upon the rules of what they can
access.
\begin{quote}\begin{description}
\item[{Return type}] \leavevmode
\code{QuerySet}

\end{description}\end{quote}

\end{fulllineitems}


\end{fulllineitems}

\index{Tbluser (class in timetracker.tracker.models)}

\begin{fulllineitems}
\phantomsection\label{timetracker:timetracker.tracker.models.Tbluser}\pysiglinewithargsret{\strong{class }\code{timetracker.tracker.models.}\bfcode{Tbluser}}{\emph{*args}, \emph{**kwargs}}{}
Models the user table and provides the admin interface with the
niceties it needs.

This model is the central pillar to this entire application.

The permissions for a user is determined in the view functions, not in a
table this is a design choice because there is only a minimal set of
things to have permissions \emph{over}, so it would be overkill to take full
advantage of the MVC pattern.

User

The most general and base type of a User is the \emph{RUSER}, which is
shorthand (and what actually gets stored in the database) for Regular
User. A regular user will only be able to access a specific section of the
site.

Team Leader

The second type of User is the \emph{TEAML}, this user has very similar level
access as the administrator type but has only a limited subset of their
access rights. They cannot have a team of their own, but view the team
their manager is assigned. They cannot view and/or change job codes, but
they can create new users with all the \emph{other} information that they
need. They can view/create/add/change holidays of themselves and the users
that are assigned to their manager.

Administrator

The third type of User is the Administrator/\emph{ADMIN}. They have full access to all
functions of the app. They can view/create/change/delete members of their
team. They can view/create/add/change holidays of all members of their
team and themselves. They can create users of any type.
\index{display\_user\_type() (timetracker.tracker.models.Tbluser method)}

\begin{fulllineitems}
\phantomsection\label{timetracker:timetracker.tracker.models.Tbluser.display_user_type}\pysiglinewithargsret{\bfcode{display\_user\_type}}{}{}
Function for displaying the user\_type in admin.
\begin{quote}\begin{description}
\item[{Note }] \leavevmode
This method shouldn't be called directly.

\item[{Return type}] \leavevmode
\code{string}

\end{description}\end{quote}

\end{fulllineitems}

\index{get\_administrator() (timetracker.tracker.models.Tbluser method)}

\begin{fulllineitems}
\phantomsection\label{timetracker:timetracker.tracker.models.Tbluser.get_administrator}\pysiglinewithargsret{\bfcode{get\_administrator}}{}{}
Returns the {\hyperref[timetracker:timetracker.tracker.models.Tbluser]{\code{Tbluser}}} who is this instances Authorization link
\begin{quote}\begin{description}
\item[{Returns}] \leavevmode
A {\hyperref[timetracker:timetracker.tracker.models.Tbluser]{\code{Tbluser}}} instance

\item[{Return type}] \leavevmode
{\hyperref[timetracker:timetracker.tracker.models.Tbluser]{\code{Tbluser}}}

\end{description}\end{quote}

\end{fulllineitems}

\index{get\_holiday\_balance() (timetracker.tracker.models.Tbluser method)}

\begin{fulllineitems}
\phantomsection\label{timetracker:timetracker.tracker.models.Tbluser.get_holiday_balance}\pysiglinewithargsret{\bfcode{get\_holiday\_balance}}{\emph{year}}{}
Calculates the holiday balance for the employee

This method loops over all {\hyperref[timetracker:timetracker.tracker.models.TrackingEntry]{\code{TrackingEntry}}} entries attached to
the user instance which are in the year passed in, taking each entries
day\_type and looking that up in a value map.

Values can be:
\begin{enumerate}
\item {} 
Holiday: Remove a day

\item {} 
Work on Public Holiday: Add two days

\end{enumerate}
\begin{enumerate}
\setcounter{enumi}{1}
\item {} 
Return for working Public Holiday: Remove a day

\end{enumerate}
\begin{quote}\begin{description}
\item[{Parameters}] \leavevmode
\textbf{year} (\code{int}) -- The year in which the holiday balance should be
calculated from

\item[{Return type}] \leavevmode
\code{Integer}

\end{description}\end{quote}

\end{fulllineitems}

\index{get\_total\_balance() (timetracker.tracker.models.Tbluser method)}

\begin{fulllineitems}
\phantomsection\label{timetracker:timetracker.tracker.models.Tbluser.get_total_balance}\pysiglinewithargsret{\bfcode{get\_total\_balance}}{\emph{ret='html'}}{}
Calculates the total balance for the user.

This method iterates through every {\hyperref[timetracker:timetracker.tracker.models.TrackingEntry]{\code{TrackingEntry}}} attached to
this user instance which is a working day, multiplies the user's
shiftlength by the number of days and finds the difference between the
projected working hours and the actual working hours.

The return type of this function is different depending on the
argument supplied.
\begin{quote}\begin{description}
\item[{Note }] \leavevmode
To customize how the CSS class is determined when using the
html mode you will need to change the ranges in the
tracking\_class\_map attribute.

\item[{Parameters}] \leavevmode
\textbf{ret} -- Determines the return type of the function. If it is not
supplied then it defaults to `html'. If it is `int' then
the function will return an integer, finally, if the
string `dbg' is passed then we output all the values used
to calculate and the final value.

\item[{Return type}] \leavevmode
\code{string} or \code{integer}

\end{description}\end{quote}

\end{fulllineitems}

\index{is\_admin() (timetracker.tracker.models.Tbluser method)}

\begin{fulllineitems}
\phantomsection\label{timetracker:timetracker.tracker.models.Tbluser.is_admin}\pysiglinewithargsret{\bfcode{is\_admin}}{}{}
Returns whether or not the user instance is an admin type user. The
two types of `admin'-y user\_types are ADMIN and TEAML.
\begin{quote}\begin{description}
\item[{Return type}] \leavevmode
\code{boolean}

\end{description}\end{quote}

\end{fulllineitems}

\index{name() (timetracker.tracker.models.Tbluser method)}

\begin{fulllineitems}
\phantomsection\label{timetracker:timetracker.tracker.models.Tbluser.name}\pysiglinewithargsret{\bfcode{name}}{}{}
Utility method for returning users full name. This is useful for when
we are pretty printing users and their names. For example in e-mails
and or when we are displaying users on the front-end.
\begin{quote}\begin{description}
\item[{Return type}] \leavevmode
\code{string}

\end{description}\end{quote}

\end{fulllineitems}

\index{tracking\_entries() (timetracker.tracker.models.Tbluser method)}

\begin{fulllineitems}
\phantomsection\label{timetracker:timetracker.tracker.models.Tbluser.tracking_entries}\pysiglinewithargsret{\bfcode{tracking\_entries}}{\emph{year=2012}, \emph{month=5}}{}
Returns all the tracking entries associated with
this user.

This is particularly useful when required to make a report or generate
a specific view of the tracking entries of the user.
\begin{quote}\begin{description}
\item[{Parameters}] \leavevmode\begin{itemize}
\item {} 
\textbf{year} -- The year in which the QuerySet should be filtered
by. Defaults to the current year.

\item {} 
\textbf{month} -- The month in which the QuerySet should be filtered
by. Defaults to the current month.

\end{itemize}

\item[{Return type}] \leavevmode
\code{QuerySet}

\end{description}\end{quote}

\end{fulllineitems}


\end{fulllineitems}

\index{TrackingEntry (class in timetracker.tracker.models)}

\begin{fulllineitems}
\phantomsection\label{timetracker:timetracker.tracker.models.TrackingEntry}\pysiglinewithargsret{\strong{class }\code{timetracker.tracker.models.}\bfcode{TrackingEntry}}{\emph{*args}, \emph{**kwargs}}{}
Model which is used to enter working logs into the database.

A tracking entry consists of several fields:-
\begin{enumerate}
\item {} 
Entry date: The date that the working log happened.

\item {} 
Start Time: The start time of the working day.

\item {} 
End Time: The end time of the working day.

\item {} 
Breaks: Any breaks taken during that day.

\item {} 
Day Type: The type of working log.

\end{enumerate}

Again, the TrackingEntry model is a core component of the time tracking
application. It directly links users with the time-spent at work and the
the type of day that was.

\end{fulllineitems}



\subsection{timetracker.tracker.admin}
\label{timetracker:timetracker-tracker-admin}\label{timetracker:module-timetracker.tracker.admin}\index{timetracker.tracker.admin (module)}
This module directly deals with how models are interacted with in the admin
interface. This has no direct impact for users but it is useful for webmasters
administrating the application.
\index{AuthAdmin (class in timetracker.tracker.admin)}

\begin{fulllineitems}
\phantomsection\label{timetracker:timetracker.tracker.admin.AuthAdmin}\pysiglinewithargsret{\strong{class }\code{timetracker.tracker.admin.}\bfcode{AuthAdmin}}{\emph{model}, \emph{admin\_site}}{}
Creates access to and customizes the admin interface to the
Tblauthorization instances. We add the \_\_unicode\_\_ and the display\_users
functions so that the display allows us to view the team associated with
the administrator and the administrator's printed representation.

\end{fulllineitems}

\index{TrackerAdmin (class in timetracker.tracker.admin)}

\begin{fulllineitems}
\phantomsection\label{timetracker:timetracker.tracker.admin.TrackerAdmin}\pysiglinewithargsret{\strong{class }\code{timetracker.tracker.admin.}\bfcode{TrackerAdmin}}{\emph{model}, \emph{admin\_site}}{}
Creates access to and customizes the admin interface to the
TrackingEntry instances. We have no special functions or list\_display
additions because the default values are useful enough as the interface to
edit these items is far more useful and better programmed than the basic
model editor the admin interface provides.

\end{fulllineitems}

\index{UserAdmin (class in timetracker.tracker.admin)}

\begin{fulllineitems}
\phantomsection\label{timetracker:timetracker.tracker.admin.UserAdmin}\pysiglinewithargsret{\strong{class }\code{timetracker.tracker.admin.}\bfcode{UserAdmin}}{\emph{model}, \emph{admin\_site}}{}
Creates access to and customizes the admin interface to the tbluser
instances. We give the list\_display of \_\_unicode\_\_ and a 2nd type of
display\_user\_tyep because this shows the printed representation of these
values and makes it easier to navigate a large selection of users.

actions is a list of functions which are in the `Actions' list, there is a
default value of `delete selected \textless{}table\textgreater{}' which Django inserts
automatically. We add here the send\_password\_reminder function defined
above.

\end{fulllineitems}

\index{send\_password\_reminder() (in module timetracker.tracker.admin)}

\begin{fulllineitems}
\phantomsection\label{timetracker:timetracker.tracker.admin.send_password_reminder}\pysiglinewithargsret{\code{timetracker.tracker.admin.}\bfcode{send\_password\_reminder}}{\emph{modeladmin}, \emph{request}, \emph{queryset}}{}
Send an e-mail reminder to all the selected employees.

This appears as an option in the `Action' list in the admin interface for
when editing the tbluser instances. This allows you to send an e-mail
reminder en masse to all users selected.

\end{fulllineitems}



\subsection{timetracker.tracker.forms}
\label{timetracker:timetracker-tracker-forms}\label{timetracker:module-timetracker.tracker.forms}\index{timetracker.tracker.forms (module)}
Forms used for user input for the tracker app

Several forms are extremely simple and thus it's not required to `hard-code'
them into HTML templates, which would couple our output with the view
functions which simply isn't good MVC practice.


\subsubsection{Module Overview}
\label{timetracker:module-overview}
\begin{tabulary}{\linewidth}{|L|L|}
\hline
\textbf{} & \textbf{}\\\hline

{\hyperref[timetracker:timetracker.tracker.forms.EntryForm]{\code{EntryForm}}}
 & 
{\hyperref[timetracker:timetracker.tracker.forms.AddForm]{\code{AddForm}}}
\\\hline

{\hyperref[timetracker:timetracker.tracker.forms.Login]{\code{Login}}}
 & \\\hline
\end{tabulary}

\index{AddForm (class in timetracker.tracker.forms)}

\begin{fulllineitems}
\phantomsection\label{timetracker:timetracker.tracker.forms.AddForm}\pysiglinewithargsret{\strong{class }\code{timetracker.tracker.forms.}\bfcode{AddForm}}{\emph{data=None}, \emph{files=None}, \emph{auto\_id='id\_\%s'}, \emph{prefix=None}, \emph{initial=None}, \emph{error\_class=\textless{}class `django.forms.util.ErrorList'\textgreater{}}, \emph{label\_suffix=':'}, \emph{empty\_permitted=False}}{}
Add entry form

This Class creates a form which allows users to add an entry into the
timetracking portion of the app. See \code{ChangeEntry} for a detailed
description of these two classes as they have a high coupling factor.

\end{fulllineitems}

\index{EntryForm (class in timetracker.tracker.forms)}

\begin{fulllineitems}
\phantomsection\label{timetracker:timetracker.tracker.forms.EntryForm}\pysiglinewithargsret{\strong{class }\code{timetracker.tracker.forms.}\bfcode{EntryForm}}{\emph{data=None}, \emph{files=None}, \emph{auto\_id='id\_\%s'}, \emph{prefix=None}, \emph{initial=None}, \emph{error\_class=\textless{}class `django.forms.util.ErrorList'\textgreater{}}, \emph{label\_suffix=':'}, \emph{empty\_permitted=False}}{}
Change entry form

This Class creates a form which allows users to change an existing entry
which they have added to the timetracking portion of the site. The fields
match up precisely to the {\hyperref[timetracker:timetracker.tracker.forms.AddForm]{\code{AddForm}}} because the data will be the
same. We have duplicated the code here because we need to explicitly set
the id values using the widget.attrs.update on the widget dict. This is so
that each widget could be styled individually and so that we have
something easily accessbile by front-end javascript code.

\end{fulllineitems}

\index{Login (class in timetracker.tracker.forms)}

\begin{fulllineitems}
\phantomsection\label{timetracker:timetracker.tracker.forms.Login}\pysiglinewithargsret{\strong{class }\code{timetracker.tracker.forms.}\bfcode{Login}}{\emph{data=None}, \emph{files=None}, \emph{auto\_id='id\_\%s'}, \emph{prefix=None}, \emph{initial=None}, \emph{error\_class=\textless{}class `django.forms.util.ErrorList'\textgreater{}}, \emph{label\_suffix=':'}, \emph{empty\_permitted=False}}{}
Basic login form

This form renders into a very simple login field, with fields identified
differently, both for styling and the optional javascript by-name
handling. Whilst this form is extremely simple, coding one each and
every time we wish to use one is just dumb and it's much easier to code it
once and import it into our project.

\end{fulllineitems}



\section{Client-side Javascript modules}
\label{javascript-files:client-side-javascript-modules}\label{javascript-files::doc}

\subsection{Calendar Module}
\label{javascript-files:calendar-module}\index{ajaxCall() (built-in function)}

\begin{fulllineitems}
\phantomsection\label{javascript-files:ajaxCall}\pysiglinewithargsret{\bfcode{ajaxCall}}{\emph{form}}{}
\end{fulllineitems}


Creates an ajax call depending on what called the function. Server-side there
is a view at domain/ajax/ which is designed to intercept all ajax calls.

The idea is that you define a function, add it to the ajax view's dict of
functions along with a tag denoting it's name, and then pass the string to
the `form\_type' json you sent to that view.

In this particular ajax request function we're pulling out form data
depending on what form calls the ajaxCall.
\begin{quote}
\begin{quote}\begin{description}
\item[{paramemter form}] \leavevmode
This argument is the string identifier of the form from
which you wish to send the data from. The possible
choices are the Add Form and the Change Form.

\item[{returns}] \leavevmode
This function returns false so that the form doesn't try to
carry on with it's original function.

\end{description}\end{quote}
\end{quote}
\index{onOptionChange() (built-in function)}

\begin{fulllineitems}
\phantomsection\label{javascript-files:onOptionChange}\pysiglinewithargsret{\bfcode{onOptionChange}}{\emph{element}}{}
\end{fulllineitems}


Specific selections determine which elements of the form are disabled. For
example there is no need to allow people to change their working time for a
vacation day. Similarly, if they have previously selected a vacation day, then
we need to re-enable the form else they will no longer be able to enter the
time into the fields.
\begin{quote}
\begin{quote}\begin{description}
\item[{paramemter element}] \leavevmode
This argument is the string identifier of the form
from which you wish to send the data from. The
possible choices are the Add Form and the Change
Form.

\item[{returns}] \leavevmode
True. This always returns true to signify to any programmatic
callers that we have finished the function. There are no error
codes or errors thrown.

\end{description}\end{quote}
\end{quote}


\chapter{Indices and tables}
\label{index:indices-and-tables}\begin{itemize}
\item {} 
\emph{genindex}

\item {} 
\emph{modindex}

\item {} 
\emph{search}

\end{itemize}


\renewcommand{\indexname}{Python Module Index}
\begin{theindex}
\def\bigletter#1{{\Large\sffamily#1}\nopagebreak\vspace{1mm}}
\bigletter{t}
\item {\texttt{timetracker.apache\_settings}}, \pageref{settings:module-timetracker.apache_settings}
\item {\texttt{timetracker.settings}}, \pageref{settings:module-timetracker.settings}
\item {\texttt{timetracker.tracker.admin}}, \pageref{timetracker:module-timetracker.tracker.admin}
\item {\texttt{timetracker.tracker.forms}}, \pageref{timetracker:module-timetracker.tracker.forms}
\item {\texttt{timetracker.tracker.models}}, \pageref{timetracker:module-timetracker.tracker.models}
\item {\texttt{timetracker.utils.calendar\_utils}}, \pageref{timetracker:module-timetracker.utils.calendar_utils}
\item {\texttt{timetracker.utils.datemaps}}, \pageref{timetracker:module-timetracker.utils.datemaps}
\item {\texttt{timetracker.utils.decorators}}, \pageref{timetracker:module-timetracker.utils.decorators}
\item {\texttt{timetracker.utils.error\_codes}}, \pageref{timetracker:module-timetracker.utils.error_codes}
\item {\texttt{timetracker.views}}, \pageref{timetracker:module-timetracker.views}
\end{theindex}

\renewcommand{\indexname}{Index}
\printindex
\end{document}
